\begin{center}
    \section{Предисловие}
\end{center}

Я прекрасно помню тот момент, когда перед началом второго семестра третьего курса узнал, что меня ждёт годовой курс квантовой механики. И хоть за моими плечами уже был один курс теоретической физики, а именно теории поля, квантовая механика оставалась для меня оплотом ``новой'' физики. Казалось, если получится понять квантовую механику, то тебе под силу понять всё. Тогда я даже не представлял, как сильно моя заинтересованность этой сферой изменит мой жизненный путь.

Семинары по квантовой механике, наверное, единственные семинары, которые я не пропустил ни разу. Сидя на них, я усиленно старался вникнуть в новый математический аппарат и осознать новые постулаты. Получалось это далеко не всегда. Поэтому, приходя в общежитие, я с большим интересом открывал различные источники, пытаясь разобраться, что же в этом квантовом мире происходит. Не раз я находил себя сидящим за столом, с 20 открытыми на ноутбуке вкладками и лежащим передо мной третьим томиком Ландау -- Лифшица. Нотация Дирака, волновая функция, туннелирование, эрмитовы операторы, уравнение Шрёдингера -- все эти новые для меня объекты кружили надо мной, пока я пытался их осознать и почувствовать. Теория давалась мне не быстро, но зато достаточно стабильно. С практикой ситуация была совсем иная.

Каждый раз открывая задавальник и обещая себе решить все задачи самостоятельно, через пару часов я кидал на стол ручку и лез открывать задачник Белоусова или искать ответы в интернете. Нельзя сказать, что я был совсем безнадёжен: все упражнения (читай как упрощённые задачи) и пару-тройку задач я всё-таки решил сам. Однако большинство задач казались мне неподъёмными. Я видел единственный вариант их решения -- подсмотреть где-то идею и дальше решать самостоятельно. Это действительно работало, но не приносило мне никакого удовольствия, так как создавалось ощущение, что я ничего не понимаю. И хоть на сдачах у меня получалось отвечать достаточно хорошо, я не чувствовал в себе уверенность, что смог бы сам решить не типовую задачу, если бы встретился с ней в ``реальной'' жизни.

Второй семестр только усугубил ситуацию. Мой семинарист отличался глубоким уровнем понимания предмета и тотальным неумением его объяснять (возможно ещё и нежеланием). Поэтому, много задач и тем второго семестра легли на плечи самих студентов. Простыми назвать их было совсем нельзя. С релятивизмом и матрицами Дирака я смог разобраться только спустя два года. Усугубляло ситуацию то, что материал по задачам было найти гораздо сложнее. Приходилось действительно разбираться, либо забивать и пользоваться решениями старшекурсников.

Вспоминая это сейчас, кажется странным, что я не потерял веру в фундаментальность и глубину квантовой механики, продолжая набивать шишки о неподъёмные задачи. Возможно, мне помогло то, что ещё на этапе изучения квантмеха я начал работать в этой сфере. Будучи окружённым разбирающимися людьми, я был замотивирован стать таким же. И именно в момент окончания курса квантмеха мне захотелось рассказывать его так, чтобы у студентов не возникало тех ощущений, которые переживал я. Я искренне верил, что рассказать квантмех без боли можно.

Собственно, так и появились сначала мои семинары, а затем и эти конспекты. Всё, что вы будете читать далее – это моя попытка облегчить ваш путь в понимании квантовой механики и решении задач. Здесь подробно решена большая часть задач, которые я решал во время своего курса по квантовой механике. Однако, помимо задач, здесь собрано достаточно последовательное (на мой взгляд) построение теории. Хоть теория может показаться недостаточно глубокой, её достаточно для осознания и самостоятельного решения любой из представленных задач. Если у читателя получится решить хотя бы несколько задач, используя только теоретические выкладки и знания математики на уровне третьекурсника, то моя работа была проделана не зря.

Конечно, эти конспекты и для меня самого. Пока они создавались, я систематизировал все свои знания и даже приобрёл новые. Всего есть три основных источника, которыми я пользовался при создании этих семинаров. Основная книга, которой я пользовался при создании семинаров, это книга \textit{А. Львовского ``Отличная квантовая механика''} \cite{ОКМ}. Именно она вдохновила меня на то, что квантовую механику можно рассказывать понятно, и что рассказывать её надо через задачки и упражнения. Поэтому в этих семинарах вы найдёте с ней много сходств. Самое большое различие, однако, заключается в том, что я ориентировался на программу и знания третьекурсника МФТИ (если быть точнее, то на студента родного для меня факультета аэрокосмических исследований). Поэтому у меня получилось изложить материал в более коротком издании, при этом решая более сложные задачи.

Вторая и третья книги более академичные -- это \textit{Л. Д. Ландау, Е. М. Лифшиц ``Теоретическая физика, том 3. Квантовая механика''} \cite{ЛЛ} и \textit{David J. Griffiths ``Introduction to Quantum Mechanics''} \cite{IQM}, соответственно. Подход в них, конечно, более строгий, однако читаются они, на удивление, достаточно плавно. В ЛЛ вы найдёте более математический подход, у Гриффитса же можно найти много интересных задач и достаточно подробное решение к ним. Если бы меня сейчас спросили, по какому из этих учебников лучше ботать -- я бы точно ответил, что по учебнику Гриффитса. Поэтому, если хорошо читаете по английский, очень вам его советую.

Не обошлось и без задачника Белоусова по теоретической физике \cite{Белоусов}. Задач в нём решено много, и большинство из них из года в год попадаются в программе ФАКТ-а. Все решения, взятые оттуда, я попытался расписать чуть более подробно, для формирования у читателя более полной картины.

То, что вы это читаете, для меня уже большое достижение. Я буду искренне рад, если смогу помочь вам сдать экзамен, сделать домашку или, может даже, заинтересовать вас в такой непростой, но очень многогранной и красивой науке, как квантовая механика. Если при прочтении у вас появится желание поблагодарить меня или вы нашли опечатку и хотите её исправить -- заходите в GitHub репозиторий \url{https://github.com/MVanza/QM-handbook-for-mipt}, ставьте звездочку и делайте pull request (или пишите мне в телеграм, ссылка там есть). И, конечно, стоит упомянуть, что один с таким количеством текста, формул и рисунков я бы не справился. Поэтому, в конце есть важная часть с благодарностями.
\vspace{1em}
\begin{flushright}
    \textit{Татаркин Олег,}\\
    \textit{2024}
\end{flushright}
