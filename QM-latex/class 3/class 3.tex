\begin{center}
    \section{Семинар III}
\end{center}
\subsection{Эволюция квантовых систем. Уравнение Шрёдингера}
Перед тем, как начать обсуждать динамику квантовых систем, важно обговорить одно исключение: время в квантовой механике не рассматривается как оператор. Не существует ни собственных состояний времени, ни квантового времени. Время – просто непрерывная переменная.

Теперь перейдём к эволюции квантовых объектов. Поставим задачу: при заданном начальном состоянии $\ket{\psi(0)}$ системы нужно определить её состояние $\ket{\psi(t)}$ в произвольный момент времени. Решение этой задачи не получится вывести из аппарата, который у нас есть на данном этапе, поэтому постулируем, что эволюция системы будет описываться следующим уравнением:
\[
\ket{\psi_E(t)} = \ket{\psi(0)}e^{-\frac{i}{\hbar}Et}.
\]
Это уравнение имеет место, когда система находится в состоянии с определенной энергией.

В классической механике для описания динамики системы достаточно было знать \textit{гамильтониан} системы. В квантовой механике всё работает примерно так же, только гамильтониан надевает шляпку и становится оператором $\hat{H}$. Как и у любого оператора, у гамильтониана есть свои собственные состояния и собственные значения. Логично предположить, что эти значения и состояния будут соответствовать определенным значениями энергии:
\[
\hat{H} = \sum_j E_j\ket{E_j}\bra{E_j}.
\]
Тогда, любое состояние может быть разложено в этом базисе:
\[
\ket{\psi(0)} = \sum_j \psi_j\ket{E_j}.
\]
Собственные энергетические состояния называют \textit{стационарными}. Тогда, исходя из нашего постулата, эволюция системы будет описываться уравнением
\[
\ket{\psi(t)} = \sum_j \psi_j\ket{E_j}e^{-\frac{i}{\hbar}E_jt}.
\]
С помощью этого уравнения можно непосредственно вычислять эволюции состояния. Однако на практике мы чаще будем пользоваться двумя другими уравнениями, о которых сейчас поговорим.

Мы уже привыкли пользоваться операторами для изменения состояния системы. Давайте и здесь введём унитарный оператор, который назовём оператором эволюции. Он будет удовлетворять уравнению:
\[
\ket{\psi_E(t)} = \hat{U}\ket{\psi(0)}
\]
В явном виде матричный элемент оператора эволюции выглядит как
\[
U_{jk} = \bra{E_j}\hat{U}\ket{E_k} = e^{-\frac{i}{\hbar}E_kt}\bra{E_j}\ket{E_k} = e^{-\frac{i}{\hbar}E_kt}\delta_{jk}.
\]
Или, если записать это в бракет нотации
\[
\hat{U} = \sum_{k}e^{-\frac{i}{\hbar}E_kt}\ket{E_k}\bra{E_k}.
\]
Чтобы получить итоговый вид оператора эволюции, перепишем его через оператор гамильтониана: 
\[
\hat{U} = e^{-\frac{i}{\hbar}\hat{H}t}
\]
Чтобы получить последнее, самое важное уравнение, продифференцируем состояние, полученное применением оператора эволюции на начальное состояние:
\[
\frac{d}{dt}e^{-\frac{i}{\hbar}\hat{H}t}\ket{\psi(0)} = -\frac{i}{\hbar}\hat{H}e^{-\frac{i}{\hbar}\hat{H}t}\ket{\psi(0)} = -\frac{i}{\hbar}\hat{H}\ket{\psi(t)}.
\]
Если оставить справа только оператор гамильтониана, то мы получим \textit{уравнение Шрёдингера}:
\[
i\hbar\frac{d}{dt}\ket{\psi(t)} = \hat{H}\ket{\psi(t)}.
\]
В общем случае гамильтониан также может зависеть от времени, но этот случай мы рассмотрим позднее. Пока наша задача в том, чтобы найти множество энергетических собственных значений и состояний гамильтониана. Т.е. нам нужно найти состояния $\ket{\psi}$, такие что
\[
\hat{H}\ket{\psi} = E\ket{\psi}.
\]
Это уравнение называется \textit{стационарным уравнением Шрёдингера}. Так, для гамильтониана частицы в поле $\hat{H} = V(\hat{x}) + \frac{\hat{p}^2}{2M}$, где первый член -- это потенциальная энергия, а второй -- кинетическая, стационарное уравнение Шрёдингера будет выглядеть следующим образом:
\[
\left[V(x) - \frac{\hbar^2}{2M}\frac{d^2}{dx^2}\right]\psi(x) = E\psi(x).
\]
Здесь мы перешли от вектора состоянии к волновой функции, домножив слева на бра состояние $\bra{x}$.

Можно сказать, что теорминимум нами был освоен и можно приступать к решению первых содержательных задач. В целом, решение большинства задач на курсе по квантовой механике сводится к решению либо стационарного, либо зависящего от времени уравнения Шрёдингера. Поэтому запоминаем его, записываем на обложках тетрадей, делаем татуировки и т.п. Чтобы помочь вам запомнить, решим несколько задач на потенциальные ямы. Этот тип задач основан на том, что мы решаем стационарное уравнение Шрёдингера, но с разным потенциалом. В общем, хватит слов, переходим к задаче.
\excersize{Упражнение №12}{darklavender}
\begin{center}
    \textit{Частица массы m совершает финитное движение в одномерной ``прямоугольной'' потенциальной яме конечной глубины:}
    \[
    U(x) = 
    \begin{cases}
    -U_0,& |x| < a,\\
    0, & |x| > a.
    \end{cases}
    \]
    \textit{Найдите уровни энергии $E_n$ и волновые функции $\psi_n(x)$ стационарных состояний. Исследуйте, существуют ли связанные состояния в ``прямоугольной'' потенциальной яме фиксированной ширины $2a$, если $U_0 \rightarrow 0$?}
\end{center}
Для начала разберёмся с условием. \textit{Связанные состояния} – состояния, для которых волновая функция при $x \rightarrow \pm\infty$ равна 0. Получается состояние как бы локализовано в определенном объёме пространства. Из-за наложенных ограничений связанные состояния имеют \textit{дискретный спектр} собственных значений энергии, которые называют энергетическими уровнями.

На рисунке \ref{fig 3.1} можно посмотреть, как выглядит яма из задачи. Думаю, уже сейчас понятно, насколько это модельная задача. Но для понимания поведения частиц, хорошо описываемых уравнением Шрёдингера, она незаменима.
\begin{figure}[h!]
\centering
\includegraphics[scale=0.3]{class 3/images/hole.png}
\caption{Потенциальная симметричная прямоугольная яма}
\label{fig 3.1}
\end{figure}\\
Выпишем одномерное стационарное уравнение Шрёдингера:
\[
U(x)\psi(x) - \frac{\hbar^2}{2m}\psi''(x) = E\psi(x).
\]
Перепишем его, чтобы в правой части остался только 0:
\[
\psi''(x) + \frac{2m}{\hbar^2}(E-U)\psi(x) = 0
\]
Теперь давайте по отдельности рассмотрим каждую из трёх областей. В первой области (т.е. в области $x <-a$) потенциальная энергия нулевая. Так как мы рассматриваем случай финитного движения, энергия частицы меньше нуля. В классическом случае мы бы с уверенностью сказали, что в этой области частица присутствовать не может. Но в квантовой механике не всё так гладко. Давайте решим стационарное уравнение Шрёдингера для этого случая:
\[
\psi_I'' - \kappa^2\psi_I = 0, \quad \kappa^2 = \frac{2m}{\hbar^2}|E|
\]
Энергию мы берём по модулю, потому что она отрицательная, а $\kappa$ в квадрате. Это гармоническое уравнение, только с минусом. Решение такого уравнения известно и выражается в виде суммы двух экспонент с разными знаками:
\[
\psi_I(x) = Ae^{\kappa x} + \widetilde{A}e^{-\kappa x}
\]
Оказывается, что второй член из этого уравнения выпадает. Это связанно с условием нормировки -- чтобы волновая функция при $x \rightarrow -\infty$ не уходила на бесконечность. Тогда мы видим, что волновая функция для первой области имеет следующий вид:
\[
\psi_I(x) = Ae^{\kappa x},
\]
то есть функция затухает экспоненциально и на минус бесконечности уходит в ноль.

Думаю, понятно, что для третьей области решение будет аналогично. Для тренировки можете проделать все выкладки самостоятельно, я же выпишу только ответ:
\[
\psi_{III}(x) = De^{-\kappa x}
\]
Теперь выпишем уравнение для второй области:
\[
\psi_{II}'' + k^2\psi_{II} = 0, \quad k^2 = \frac{2m}{\hbar^2}(U_0 - |E|)
\]
Соответственно, решение для него
\[
\psi_{II}(x) = B\cos kx + C\sin kx
\]

Для нахождения констант воспользуемся важным теоретическим фактом о том, что если операторы коммутируют, то их собственные функции совпадают. Далее, исходя из симметрии задачи (U(x) = U(-x)), понимаем, что оператор инверсии коммутирует с гамильтонианом:
\[
[\hat{H}, \hat{I}] = [U(x) + \frac{\hat{p}^2}{2m}, \hat{I}] = [U(x), \hat{I}] + [\hat{p}^2, \hat{I}] = 0
\]
Значит, можно искать собственные функции гамильтониана как собственные функции оператора инверсии. Но мы знаем, что для оператора инверсии собственные функции обладают определенной чётностью. Тогда найдём чётные и нечётные волновые функции нашей задачи:
\[
\psi_+ =
\begin{cases}
    Ae^{\kappa x}& x < -a,\\
    B \cos kx & -a < x < a,\\
    Ae^{-\kappa x}& x > a
\end{cases}\quad
\psi_- =
\begin{cases}
    -De^{\kappa x}& x < -a,\\
    C \sin kx & -a < x < a,\\
    De^{-\kappa x}& x > a
\end{cases}
\]
Думаю, понятно, откуда взялись функции с синусом и косинусом. Чуть менее понятно может быть с оставшимися функциями. Они связаны с самим определением четной функции: $Ae^{\kappa x}$ есть у нас из решения, нам нужно составить такую функцию, которая при изменении знака икс не будет менять знак выражения. Получается $Ae^{-\kappa x}$. То же самое для второго: $De^{-\kappa x}$ есть, $-De^{\kappa x}$ мы дополняем для нечетности функции.

Рассмотрим условия на волновую функцию. Пока что она не определена на границах. То есть нужно ввести условия сшивки. Их мы берем из физического смысла волновой функции, а именно:
\[
\psi(x) - \text{непрерывна},\; \frac{\psi'(x)}{\psi(x)} - \text{непрерывна}
\]
Из этих условий ``сошьём'' границы для чётных и нечётных функций отдельно. Так как потенциал симметричен, достаточно рассмотреть только одну границу. 
\[
\frac{\psi'_+(x)}{\psi_+(x)}\bigg|_{x=a-0} = \frac{\psi'_+(x)}{\psi_+(x)}\bigg|_{x=a+0} = \frac{-Bk\sin ka}{B\cos ka} = \frac{-\kappa A e^{-\kappa a}}{Ae^{-\kappa a}} =>
\]
\[
k\tg ka = \kappa
\]
Для нечётных:
\[
\frac{\psi'_-(x)}{\psi_-(x)}\bigg|_{x=a-0} = \frac{\psi'_-(x)}{\psi_-(x)}\bigg|_{x=a+0} = \frac{Ck\cos ka}{C\sin ka} = \frac{-\kappa D e^{-\kappa a}}{De^{-\kappa a}} =>
\]
\[
k\ctg ka = -\kappa
\]
Ещё немного повертим выражение и придём к итоговому, красивому виду
\[
(\kappa a)^2 = (k_0a)^2 - (ka)^2, \; \text{где} \;k_0^2 = k^2 + \kappa^2 =>  
\]
\[
=> \kappa a = \sqrt{(k_0a)^2 - (ka)^2} => ka\tg ka = \sqrt{(k_0a)^2 - (ka)^2} =>
\]
\[
=> \tg ka = \sqrt{\frac{(k_0a)^2}{(ka)^2} - 1}
\]
\[
\tg^2 ka = \frac{1}{\cos^2 ka} - 1 =>
=> \frac{1}{\cos^2 ka} - 1 = \frac{(k_0 a)^2}{(ka)^2} - 1 => \cos^2 ka = \frac{(ka)^2}{(k_0 a)^2} =>
\]
\[
=> |\cos ka| = \frac{ka}{k_0a}, \; \tg ka>0
\]
Проделав аналогичные вычисления для нечётных функций, в итоге получаем компактное выражение
\[
\begin{cases}
   |\cos ka| = \frac{ka}{k_0a},& \tg ka>0 \\
   |\sin ka| = \frac{ka}{k_0a},& \tg ka<0 \\
\end{cases}
\]
У нас получаются трансцендентные уравнения. Т.е. просто выразить какую-либо величину из этого уравнения – сложно, или даже нереально. Мы можем только проанализировать его. Например, графически.
\begin{figure}[!h]
  \centering
  \subfloat[График для чётных функций. Выделенные области удовлетворяют условию $\tg ka > 0$]{\includegraphics[scale=0.5]{class 3/images/even.png}}
  \hfill
  \subfloat[График для нечётных функций. Выделенные области удовлетворяют условию $\tg ka < 0$]{\includegraphics[scale=0.5]{class 3/images/odd.png}}
  \caption{}
\end{figure}
\\
Что мы можем вынести из этих графиков? Для чётных функций видно, что решение существует при любом положительном $k_0a$. Так как этот параметр отвечает параметрам задачи, получается, что при любой глубине ямы у нас существует хотя бы одно чётное решение. В то же время, из графика нечётных функций достаём следующее условие существования решений: $k_0a>\frac{\pi}{2}$. Можно легко проверить, что чётные и нечётные решения чередуется. Формулы для количества решений:
\[
[\frac{k_0a}{\pi}] + 1 = N_+,\quad[\frac{k_0a}{\pi/2}] = N_-
\]
Теперь найдём коэффициенты A и B из неиспользованного условия непрерывности:
\[
\psi_+(a-0) = \psi_+(a+0) => A = Be^{\kappa a}\cos ka
\]

\[
\braket{\psi_+} = 1 => 2 * (\int\limits_0^a |B|^2\cos^2 kx dx + \int\limits_a^\infty |A|^2e^{-2\kappa(x)} dx) = 1 =>
\]
\[
 => 2 *|B|^2 (\int\limits_0^a \cos^2 kx dx + \int\limits_a^{+\infty }\cos^2 kae^{-2\kappa(x-a)} dx) = 1 =>
\]
\[
=> |B|^2 = (a + \frac{\sin 2ka}{2k} + \frac{\cos^2 ka}{\kappa})^{-1} 
\]
Проделав ту же самую процедуру с D и C, получим
\[
D = Ce^{\kappa a}\sin ka, \quad |C|^2 = (a - \frac{\sin 2ka}{2k} + \frac{\sin^2 ka}{\kappa})^{-1}
\]
Всё, основную часть мы проанализировали, волновые функции и уровни энергии получили. Оставшиеся вопросы будут обсуждены на очном семинаре и предоставляются тем, кто его пропустил, в виде \sout{наказания} упражнения.

На следующем семинаре продолжим разбирать задачи, связанные с одномерным потенциалом и уравнением Шрёдингера.
\csquare{darklavender}