\begin{center}
    \section{Семинар V}
\end{center}
\subsection{Решение задач на потенциальные барьеры}
\excersize{Упражнение №17}{darklavender}
\begin{center}
    \textit{Получите стационарное уравнение Шрёдингера в задаче 1.4 в импульсном представлении. Найдите решение этого уравнения, описывающее связанное состояние частицы. Преобразуйте найденную волновую функцию частицы в импульсном представлении в волновую функцию в координатном представлении и удостоверьтесь в правильности ответа.}
\end{center}
Вообще, задача очень нудная, но если прорешать её самостоятельно (что я вам очень советую), то можно очень хорошо разобраться с переходами между импульсными и координатными представлениями. Напомню, что в бракет форме волновое уравнение выглядит так:
\[
i\hbar \frac{d}{dt} \ket{\psi} = \hat H \ket{\psi}
\]
Уравнение Шрёдингера из него получается путем проецирования обеих частей на вектор координаты. В нашем же случае нужно спроецировать на вектор импульса. Буду писать эту часть достаточно подробно:
\[
\bra{p} i\hbar \frac{d}{dt} \ket{\psi} = \bra{p} \hat H \ket{\psi}
\]
Раскроем оператор Гамильтона как $\hat H = \frac{\hat p^2}{2m} + U(\hat x)$ и перепишем $\bra{p}\ket{\psi}$ в виде $\psi_p$
\[
i\hbar \frac{d}{dt} \psi_p = \frac{p^2}{2m}\psi_p + \bra{p}U(\hat{x})\hat{1}_p\ket{\psi} = \frac{p^2}{2m}\psi_p + \int\bra{p}U(\hat x)\ket{p'}\psi_{p'}\;dp'.
\]
Нам необходимо вычислить матричный элемент $U(\hat{x})$ в импульсном представлении. Используем единичный оператор (только на этот раз координаты), вставляя его в наше скалярное произведение:
\[
\int\bra{p}U(\hat x)\hat{1}_x\ket{p'}\psi_{p'}\;dp' = \int \bra{p}U(\hat x)\left[\int\ket{x}\bra{x}dr\right]\ket{p'} \psi_{p'}\;dp'=
\]
\[
= \int\int U(\hat x)\bra{p}\ket{x}\bra{x}\ket{p'} \psi_{p'} dx\,dp' = \frac{1}{2\pi\hbar}\int\int U(x) e^{-\frac{i}{h}(p-p')x}\psi_{p'} dx\,dp'
\]
Подставляя его в изначальное уравнение, получим уравнение Шрёдингера в импульсном представлении:
\[
i\hbar \frac{d}{dt} \psi_p =\frac{p^2}{2m}\psi_p + \frac{1}{2\pi\hbar}\int\int U(x) e^{-\frac{i}{h}(p-p')x}\psi_{p'} dx\,dp'.
\]
Подставляя потенциальную энергию из задачи 1.4, получим:
\[
\frac{p^2}{2m}\psi_p - \frac{\kappa_0\hbar}{2\pi m}\int\int \delta(x) e^{-\frac{i}{h}(p-p')x}\psi_{p'} dx\,dp'= \frac{p^2}{2m}\psi_p - \frac{\kappa_0\hbar}{2\pi m}\int\psi_{p'} dp' =  E\psi_p
\]
Предположим, что интеграл сходится, и введём для него обозначение $I = \int \psi_{p'}dp'$. Далее, выражая энергию через $\kappa$, получим простое алгебраическое выражение на $\psi_p$:
\[
\frac{p^2}{2m}\psi_p - I\frac{\kappa_0\hbar}{2\pi m} = -\frac{\hbar^2\kappa^2}{2m}\psi_p => \psi_p = \frac{\hbar\kappa_0}{\pi}\frac{I}{p^2 + \hbar^2\kappa^2}
\]
Теперь выразим $\psi_p$ обратно через $I$ и получим:
\[
I = \frac{\hbar\kappa_0}{\pi}\int\frac{I}{p^2 + \hbar^2\kappa^2}dp = \frac{\hbar \kappa_0}{\pi}\frac{I \pi}{\hbar\kappa^2} = I\frac{\kappa_0}{\kappa} => \kappa_0 = \kappa
\]
Тогда, можем записать энергию как 
\[
E = -\frac{\kappa^2_0\hbar^2}{2m}
\]
Волновую функцию в импульсном представлении можно найти двумя путями: либо решить ранее полученное алгебраическое уравнение относительно $I$ с условием нормировки $\int|\psi_p|^2dp = 1$, либо спроецировав уже известное нам решение из задачи 1.4 $\psi(x) =\sqrt{\kappa_0}e^{-\kappa_0|x|}$ на вектор импульса, то есть посчитать скалярное произведение $\bra{p}\ket{\psi}$ (не забудьте воспользоваться единичным проекционным оператором). Удачи!
\csquare{darklavender}

\excersize{Упражнение №18}{darklavender}
\begin{center}
    \textit{Частица массы m свободно движется вдоль оси $x$ с энергий $E$ и в области $x>0$ попадает в область действия потенциала, который имеет вид:}
    
    \textit{а) прямоугольного потенциального барьера ширины $a$ и глубины $U_0$,}

    \textit{б) прямоугольного потенциального барьера ширины $a$ и высоты $U_0$. Найдите коэффициенты прохождения $T(E)$ и отражения $R(E)$ частицы от указанных потенциалов. Определите значения энергии, при которых ямы и барьеры полностью прозрачны для падающих частиц. Чем определяются эти энергии?}

    \textit{В случае потенциальной ямы предложите способ оценки энергии $E_0$, выше которой квантовые ответы практически совпадают с классическими. Какова эта энергия при прохождении электрона сквозь наноскопический слой металла: $a=10$ нм и $U_0 = 10$ эВ?}
\end{center}

Запишем стационарное уравнение Шрёдингера и уравнения для волновых функций этой задачи:
\[
\begin{cases}
\hat H\psi=E\psi \\
\psi(x\rightarrow -\infty) = e^{i\kappa x} + re^{-i\kappa x} \\
\psi(x \rightarrow +\infty) = te^{i\kappa x}
\end{cases}
\]
Первый член в левой части второго уравнения – волна до взаимодействия с барьером. Второй – волна после отражения от барьера. Соответственно, левая часть третьего уравнения отвечает за прошедшую волну. Давайте разберёмся с коэффициентами. Для этого запишем уравнение для области в потенциальной яме:
\[
\psi'' + k^2\psi = 0, \quad k^2 = \frac{2m}{\hbar^2}(|U_0| + E)
\psi(x) = c_1e^{ikx} + c_2e^{-ikx}
\]
Отмечу, что мы поменяли обозначения $k$, так как теперь рассматриваем положительные энергии. Воспользуемся условиями непрерывности ВФ и её производной и найдём коэффициенты:
\[
\begin{cases}
    \psi_I(0) = \psi_{II}(0) \\
    \psi'_I(0) = \psi'_{II}(0) 
\end{cases}
\quad
\begin{cases}
    \psi_{II}(a) = \psi_{III}(a) \\
    \psi'_{II}(a) = \psi'_{III}(a) 
\end{cases}
\]
\[
\begin{cases}
    1 + r = c_1 + c_2 \\
    1-r = \frac{k}{\kappa}(c_1 - c_2) 
\end{cases}
\quad
\begin{cases}
    c_1e^{ika} + c_2e^{-ika} = te^{i\kappa a}\\
    c_1e^{ika} - c_2e^{-ika} = \frac{\kappa}{k}te^{i\kappa a}
\end{cases}
\]
Из правой части выражаем $c_1(t)$ и $c_2(t)$:
\[
c_1 = \frac{t}{2}e^{ia(\kappa - k)}(1 + \frac{\kappa}{k})
\]
\[
c_2 = \frac{t}{2}e^{ia(\kappa + k)}(1 - \frac{\kappa}{k})
\]
Подставим в левую часть и выразим $\frac{1}{t}$:
\[
\frac{4}{te^{i\kappa a}} = e^{-ika}\frac{(\kappa + k)^2}{k\kappa} + e^{ika}\frac{(\kappa - k)(k - \kappa)}{k\kappa}
\]
Теперь сделаем небольшое отступление. Из лекции или литературных источников мы знаем, что для волновой функции выполняется уравнение непрерывности вида $\frac{\partial \rho}{\partial t} + div \vec{j} = 0$, где ток равен $\vec{j} = \frac{i\hbar}{2m}(\psi\nabla\psi^* - \psi^*\nabla\psi)$. Пусть $\psi = |\psi|e^{i\phi}$ - какое-то комплексное число. Тогда $\vec{j} = \frac{\hbar i*i}{2m}(-|\psi|^2\nabla\phi - |\psi^2|\nabla\phi) = \frac{\hbar}{m}|\psi|^2\nabla\phi$. Но тогда, вспоминая нашу задачу, видим, что коэффициент прохождения $T$ связан с коэффициентом $t$ уравнением 
\[
T = \frac{j_{\text{прох}}}{j_{\text{пад}}} = |t|^2
\]
Для коэффициента отражения аналогично $R = |r|^2$. Возвращаясь к нашей задаче, возводим модуль $\frac{1}{t}$ в квадрат и получаем:
\[
\frac{1}{T} = 1 + \frac{U_0^2}{4E(E+|U_0|)} \sin(ak)
\]
Так как $R = 1 - T$, получим 
\[
\frac{1}{R} = -\frac{U_0^2}{4E(E+U_0)} \sin(ak)
\]
Для случая потенциального барьера б достаточно заменить $(|U_0| + E)$ на $(E-U_0)$, если $E > U_0$ и на $(U_0 - E)$, если $U_0 > E$.
\csquare{darklavender}
\excersize{Упражнение №19}{darklavender}
\begin{center}
    \textit{Частица массы m свободно движется вдоль оси $x$ с энергий $E$ и попадает в область действия $\delta$-потенциала(см. задачу 1.4). Найдите коэффициенты прохождения $T(E)$ и отражения $R(E)$ частицы.}
\end{center}
Мы уже прорешали все необходимые задачи для решения этой, поэтому здесь выпишем только основные необходимые уравнения.
\[
\begin{cases}
\hat H \psi = E \psi \\
\psi(x\rightarrow -\infty) = e^{ikx} + re^{-ikx}\\
\psi(x\rightarrow +\infty) = te^{ikx}\\
\psi(+0) = \psi(-0)\\
\psi'(+0) - \psi'(-0) = -2\kappa_0\psi(0)
\end{cases}
\]
Решая эти уравнения, получим
\[
R = \frac{\kappa_0^2}{\kappa^2+\kappa^2_0}, \quad T = \frac{\kappa^2}{\kappa^2 + \kappa^2_0}
\]
\csquare{darklavender}

\excersize{Упражнение №20}{darklavender}
\begin{center}
    \textit{Частица массы m находится в связанном состоянии в $\delta$-потенциале (см. задачу 1.4). В момент t=0 происходит мгновенное изменение параметра ямы от $\kappa_0$ до $\kappa_1$. Найдите вероятность "ионизации". Обсудите эволюцию волновой функции частицы сразу после ионизации в случае, когда $\kappa_1=0$.}
\end{center}
Расчёты в задаче не представляют никакой сложности, если понимать, что есть что. Разберём по порядку. Мгновенное изменение говорит о том, что уравнение Шрёдингера остаётся стационарным. Вероятность ионизации – вероятность вылета частицы из ямы. Теперь давайте всё покажем наглядно. Из задачи 1.4 мы знаем, что волновая функция дельта ямы есть $\psi_0(x) = \sqrt{\kappa_0}e^{-\kappa_0|x|}$. Соответственно, в момент $t=0$ волновая функция запишется в виде $\psi_1(x) = \sqrt{\kappa_1}e^{-\kappa_1 |x|}$. Найдём вероятность перехода от $\psi_0$ к $\psi_1$, используя правила Борна (т.е. через скалярное произведение):
\[
P = |\int\psi_0(x)\psi_1(x)dx|^2 =\kappa_0\kappa_1|\int e^{-(\kappa_0 + \kappa_1)|x|}dx|^2 = \kappa_0\kappa_1\frac{4}{(\kappa_0 + \kappa_1)^2}
\]
Тогда вероятность выхода, то есть ионизации, есть $1 - P$, то есть
\[
P_{exit} = 1 - \kappa_0\kappa_1\frac{4}{(\kappa_0 + \kappa_1)^2}
\]
\csquare{darklavender}

На этой задаче мы временно прощаемся с одномерными потенциалами и переходим к темам осциллятора и углового момента, которые будут обсуждаться на следующем семинаре. 