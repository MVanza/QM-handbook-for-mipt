\begin{center}
    \section{Семинар VIII}
\end{center}
\subsection{Собственные значения момента импульса.}
Предыдущий семинар мы закончили на радиальном уравнении:
\[
\left[-\frac{\hbar^2}{2Mr^2}\frac{\partial}{\partial r}\left( r^2 \frac{\partial}{\partial r}\right)  + \frac{\lambda}{2Mr^2} + V(r)\right]R(r) = ER(r).
\]

Напомню, что сейчас мы хотим привести оператор $\hat{L}^2$ к диагональному виду. Однако тут возникает проблема: собственные значения $\lambda$ в $\mathbf{Y}$ вырождены. Соответственно, просто найдя $\lambda$ мы не сможем однозначно идентифицировать для неё собственное состояние $\ket{\lambda}$, так как каждому $\lambda$ будет соответствовать линейное подпространство. Для того, чтобы решить проблему, мы добавим к нашей системе оператор, коммутирующий с оператором $\hat{L}^2$. Тогда в каждом вырожденном подпространстве $\lambda$ мы сможем определить ортонормированный собственный базис, так как у коммутирующих операторов общий набор собственных состояний. Оказывается, что очень удачно для такого решения подходит оператор $\hat{L}_z$, собственные значения которого мы обозначим через $\mu$. Тогда задача сводится к поиску $\ket{\lambda\mu}$, т.е. собственных состояний для операторов $\hat{L}^2$ и $\hat{L}_z$.
\begin{equation*}
    \begin{cases}
        \hat{L^2}\ket{\lambda\mu} = \lambda\ket{\lambda\mu}\\
        \hat{L}_z\ket{\lambda\mu} = \mu\ket{\lambda\mu}
    \end{cases}
\end{equation*}

В целом, можно получить волновые функции состояния $\ket{\lambda\mu}$ , честно решив эту систему уравнений. Однако мы пойдём более ``квантовомеханическим'' путём. Подход будет похож на тот, который мы использовали при анализе осциллятора в шестом семинаре. Давайте введём \textit{повышающий и понижающий} операторы следующим образом:
\begin{equation*}
    \begin{split}
        \hat{L}_+ = \hat{L}_x + i\hat{L}_y  \\
        \hat{L}_- = \hat{L}_x - i\hat{L}_y 
    \end{split}
\end{equation*}

Некоторые особенности этих операторов:
\begin{equation*}
    \begin{aligned}
        a)& \hat{L}_+ = \hat{L}^{\dagger}_-  \\
        b)& \left[\hat{L}_z, \, \hat{L}_{\pm}\right] = \pm\hbar \hat{L}_{\pm},\; \left[\hat{L}^2, \, \hat{L}_{\pm}\right] = 0, \; \left[\hat{L}_+, \, \hat{L}_-\right] = 2\hbar \hat{L}_z \\
        c)& \hat{L}^2 = \hat{L}_+\hat{L}_- + \hat{L}^2_z - \hbar\hat{L}_z
    \end{aligned}
\end{equation*}
Используя полученные коммутационные соотношения, покажем, что состояния $\hat{L}_{\pm}\ket{\lambda\mu}$ являются собственными для операторов $\hat{L}^2$ и $\hat{L}_z$ одновременно, с собственным значениями $\lambda$, $\mu \pm \hbar$. Действительно, подействовав оператором $\hat{L}^2$ на состояние $\hat{L}_{\pm}\ket{\lambda\mu}$, получим:
\[
\hat{L}^2\hat{L}_{\pm}\ket{\lambda\mu} = \hat{L}_{\pm}\hat{L}^2\ket{\lambda\mu} = \lambda\hat{L}_{\pm}\ket{\lambda\mu}
\]
Теперь, подействуем оператором $\hat{L}_z$:
\[
\hat{L}_z\hat{L}_{\pm}\ket{\lambda\mu} = (\hat{L}_{\pm}\hat{L}_z \pm \hbar\hat{L}_{\pm})\ket{\lambda\mu} = (\mu\hat{L}_{\pm} \pm \hbar\hat{L}_{\pm})\ket{\lambda\mu} = (\mu \pm \hbar)\hat{L}_{\pm}\ket{\lambda\mu}
\]

Таким образом мы получили то, что хотели. Далее, продолжая похожие рассуждения, как и в случае с операторами рождения и уничтожения, найдём, как действует оператор $\hat{L}_{\pm}$ на состояние $\ket{\lambda\mu}$:
\[
\hat{L}_{\pm}\ket{\lambda\mu} = \sqrt{\lambda - \mu(\mu \pm \hbar)}\ket{\lambda,\mu\pm\hbar}.
\]

В очередной раз операторы оправдывают своё название. Как видите, мы либо повышаем значение $\mu$ на константу $\hbar$, либо понижаем её на то же значение. Но что это за константа? Давайте разбираться.

Для начала заметим, что выполняется неравенство $\lambda \geq \mu^2$. В самом деле, для оператора $\hat{L}^2 - \hat{L}^2_z$ состояние $\ket{\lambda\mu}$ является собственным с собственным значением $\lambda - \mu^2$. Но, в то же время, этот оператор можно переписать как $\hat{L}^2_x + \hat{L}^2_y$ (так как $\hat{L}^2 = \hat{L}^2_x + \hat{L}^2_y + \hat{L}^2_z$). Этот оператор неотрицательный и имеет неотрицательные собственные значения. Значит, $\lambda - \mu^2 \geq 0\; => \; \lambda \geq \mu^2$.

Используя оператор повышения, мы можем поднять состояние $\ket{\lambda,\mu + j\hbar}$, где $j$ - целое неотрицательное число. Но теперь мы знаем, что $(\mu + j\hbar)^2$ не должно быть больше $\lambda$. Тогда, как и в случае с понижающим оператором, нам нужно поставить условие, что, дойдя до определенного $j = j_0$ у нас получится уравнение $\hat{L}_+\ket{\lambda, \mu + j_0\hbar} = 0$. Найдём, при каком $\lambda$ это выполняется:
\begin{multline*}
\hat{L}_{+}\ket{\lambda,\mu + j_0\hbar} = \sqrt{\lambda - (\mu+j_0\hbar)(\mu + (j_0 + 1)\hbar)}\ket{\lambda,\mu +(j_0 + 1)\hbar)} = 0\; => \\ \lambda = [\mu + j_0\hbar][\mu + \hbar(j_0 + 1)]
\end{multline*}

Аналогично, для понижающего оператора должно выполняться $\hat{L}_-\ket{\lambda,\mu - k_0\hbar} = 0$. Тогда $\lambda = [\mu - k_0\hbar][\mu-\hbar(k_0 + 1)]$ и должно выполняться уравнение:
\[
 [\mu + j_0\hbar][\mu + \hbar(j_0 + 1)]=[\mu - k_0\hbar][\mu-\hbar(k_0 + 1)]
\]
Если, для удобства обозначения мы сделаем замену $\mu + j_0\hbar = x$ и $\mu - (k_0 + 1)\hbar = y$, то уравнение примет вид:
\[
x(x+\hbar) = y(y+\hbar)
\]
Если решить это квадратное уравнение относительно y и учесть, что нам подходит только тот корень, который удовлетворяет условию $x > y$, то получим $y = -(x + \hbar)$ или 
\[
\mu - (k_0 + 1)\hbar = -\mu - (j_0 + 1)\hbar\; => \; \mu = \frac{k_0 - j_0}{2}\hbar.
\]

Определим значение $\lambda$, подставив $\mu$ в уравнение, полученное выше:
\[
\lambda = [\mu + j_0\hbar][\mu + \hbar(j_0+1)] = \frac{k_0 + j_0}{2}\left(\frac{k_0 + j_0}{2} + 1\right)\hbar^2
\]

Сейчас мы определим два важных значения, с которыми вы уже не один раз сталкивались ещё со школы. Пусть $l\equiv \frac{k_0 + j_0}{2}$ - будем называть его \textit{орбитальным квантовым числом}. Его значения лежат в множестве $\{n/2,\, n\in \mathbb{Z}_+\}$, то есть являются целочисленными или полуцелыми\footnote[1]{Отмечу, что пока не было доказано, что l - только целочисленное значение, хотя при рассмотрении атомов мы пользовались только целочисленными орбитальными числами. Объяснение этой особенности будет рассмотрено далее.} неотрицательными числами (0, 1/2, 1, ... ).

Запишем собственное значение $\lambda$  через $l$:
\[
\lambda = \hbar^2(l+1)l
\]
То же самое сделаем и с собственным значением $\mu$:
\[
\mu = (l - j_0)\hbar = (-l + k_0)\hbar
\]
Пусть $m \equiv \mu/\hbar$. Мы видим, что значение $m$ зависит от $l$ и определяется в диапазоне от $-l$ до $l$ с шагом единица. Действительно, по условию $m^2 \leq l(l+1)$. Так как $j_0$ и $k_0$ - целочисленные неотрицательные значения, то шаг может происходить только на единицу. Значит, крайние значения для m - это $l$ и $-l$, иначе неравенство нарушается. Число m мы будем называть \textit{магнитным квантовым числом}. С ним вы тоже уже встречались.

Используя новые обозначения, перепишем полученное ранее действие операторов $\hat{L}_{\pm}$:
\[
\hat{L}_{\pm}\ket{lm} = \hbar\sqrt{l(l+1) - m(m \pm 1)}\ket{l, m \pm 1} = \hbar\sqrt{(l \mp m)(l \pm m + 1)}\ket{l,m\pm 1}
\]

Давайте найдём матричные элементы всех наших новых операторов в общем виде, а затем решим упражнение для конкретного орбитального числа. Начнём с самого простого оператора -- $\hat{L}^2$. Напомню, что матричный элемент -- это конструкция вида $\bra{\psi}\hat{A}\ket{\psi'}$. В нашем случае $\bra{lm}\hat{L}^2\ket{l'm'}$. Итак
\[
\bra{lm}\hat{L}^2\ket{l'm'} = \hbar^2l'(l'+1)\bra{lm}\ket{l'm'} = \hbar^2l'(l'+1)\delta_{ll'}\delta_{mm'}
\]
Далее, для оператора $\hat{L}_z$:
\[
\bra{lm}\hat{L}_z\ket{l'm'} = \hbar m'\bra{lm}\ket{l'm'} = \hbar m'\delta_{ll'}\delta_{mm'}
\]
Для повышающего и понижающего мы уже знаем действие на состояние $\ket{lm}$, так что найти матричный элемент не составит проблемы:
\[
    \bra{lm}\hat{L}_{\pm}\ket{l'm'} = \hbar\sqrt{l'(l'+1) - m'(m' \pm 1)}\delta_{ll'}\delta_{m,m'\pm 1}
\]
Операторы $\hat{L}_x$ и $\hat{L}_y$ выражаются через понижающие и повышающие операторы по определению:
\begin{align*}
    \hat{L}_x & = \frac{\hat{L}_+ + \hat{L}_-}{2};\\
    \hat{L}_y & = \frac{\hat{L}_+ - \hat{L}_-}{2i}
\end{align*}
Тогда, подставляя их в такой форме в выражение для матричных элементов, получим:
\[
\bra{lm}\hat{L}_{x}\ket{l'm'} = \frac{\hbar}{2}\left[\sqrt{l'(l'+1) - m'(m' + 1)}\delta_{ll'}\delta_{m,m' + 1} + \sqrt{l'(l'+1) - m'(m' - 1)}\delta_{ll'}\delta_{m,m' - 1}\right]
\]
\[
\bra{lm}\hat{L}_{y}\ket{l'm'} = \frac{\hbar}{2i}\left[\sqrt{l'(l'+1) - m'(m' + 1)}\delta_{ll'}\delta_{m,m' + 1} - \sqrt{l'(l'+1) - m'(m' - 1)}\delta_{ll'}\delta_{m,m' - 1}\right]
\]
Теперь, зная все матричные элементы, выполним следующее упражнение.
\excersize{Упражнение №25}{darklavender}
\begin{center}
    \textit{Постройте матрицы операторов момента импульса $\hat{L}_x$, $\hat{L}_y$, $\hat{L}_z$, а также $\hat{L}^2$, $\hat{L}_+$ и $\hat{L}_-$ для квантовой системы с орбитальным числом $l = 1$. Как выглядят собственные вектора операторов $\hat{L}^2$ и $\hat{L}_z$?}
\end{center}
Всю ``сложную'' часть мы уже проделали, осталось только подставить значение l = 1 в полученные выше формулы:
\[
\hat{L}^2 \simeq  2\hbar^2 \left(\begin{array}{ccc} 1 & 0 & 0 \\ 0 & 1 & 0 \\ 0 & 0 & 1 \end{array} \right); \; \hat{L}_z \simeq  \hbar \left(\begin{array}{ccc} 1 & 0 & 0 \\ 0 & 0 & 0 \\ 0 & 0 & -1 \end{array} \right)
\]
В остальных операторах значения уйдут с главной диагонали, т.к. в дельтах индексы m отличаются на единицу.
\[
\hat{L}_+ \simeq  \sqrt{2}\hbar \left(\begin{array}{ccc} 0 & 1 & 0 \\ 0 & 0 & 1 \\ 0 & 0 & 0 \end{array} \right); \; \hat{L}_- \simeq  \sqrt{2}\hbar \left(\begin{array}{ccc} 0 & 0 & 0 \\ 1 & 0 & 0 \\ 0 & 1 & 0 \end{array} \right)
\]
\[
\hat{L}_x \simeq  \frac{\hbar}{\sqrt{2}} \left(\begin{array}{ccc} 0 & 1 & 0 \\ 1 & 0 & 1 \\ 0 & 1 & 0 \end{array} \right);\; \hat{L}_y \simeq  \frac{i\hbar}{\sqrt{2}} \left(\begin{array}{ccc} 0 & -1 & 0 \\ 1 & 0 & -1 \\ 0 & 1 & 0 \end{array} \right)
\]

Так как матрицы операторов $\hat{L}^2$ и $\hat{L}_z$ диагональны, найти их собственные вектора несложно -- это просто базисные вектора $ \left(\begin{array}{c} 1 \\ 0  \\ 0  \end{array} \right)$, $ \left(\begin{array}{c} 0 \\ 1 \\ 0  \end{array} \right)$ и $ \left(\begin{array}{c} 0 \\ 0  \\ 1  \end{array} \right)$. 
\csquare{darklavender}

\subsection{Собственные состояния момента импульса}
Процесс нахождение собственных функций, к сожалению, не получится свести к определению новых операторов и чисто квантово-механическому подходу. Поэтому рассуждения и поиск волновых функций я вынес в \nameref{appendix:A}. Здесь же я приведу только ответ.

Итак, волновая функция состояния $\ket{lm}$ задаётся сферическими гармониками:
\[
Y^m_l(\theta, \phi) = \mathcal{N}_l\sqrt{\frac{(l+m)!}{(l-m)!}} \frac{1}{\sin^{m}\theta}\,\frac{d^{l-m}}{d(\cos\theta)^{l-m}}\sin^{2l}\theta e^{im\phi},
\]
где $\mathcal{N}_l = (-1)^l \sqrt{\frac{2l+1}{4\pi}}\frac{1}{2^l l!}$ - коэффициент нормировки.

Вычислим явно сферические гармоники для $l=0$ и $l=1$:
\[
Y^0_0(\theta,\phi) = (-1)^0\sqrt{\frac{1}{4\pi}}\sqrt{\frac{0!}{0!}}\sin^0\theta \frac{d^0}{d(\cos\theta)^0} \sin^0 \theta e^{0} = \sqrt{\frac{1}{4\pi}}
\]
\[
Y^1_1(\theta,\phi) = -\sqrt{\frac{3}{8\pi}}\sin\theta e^{i\phi}
\]
\[
Y^0_1(\theta,\phi) = \sqrt{\frac{3}{4\pi}}\cos\theta
\]
\[
Y^{-1}_1(\theta,\phi) = \sqrt{\frac{3}{8\pi}}\sin\theta e^{-i\phi}
\]

Обсудим вопрос, почему l может быть только целым. Действительно, если l может принимать полуцелые значения, значит m также может быть полуцелым. Тогда, множитель $e^{im\phi}$ даст нам ситуацию, в которой $\psi(r, \theta, \phi) = -\psi(r, \theta, \phi + 2\pi)$. А это невозможно, так как мы работаем в координатном базисе. Значит, l в радиально-симметричном поле должна принимать только целые значения.

\subsection*{Спин.}
Чтобы рассказать про спин подробно, нужно уделять этому целую лекцию (а то и две), поэтому здесь мы ограничимся самым необходимым. Если хотите узнать про спин больше, прочитайте \nameref{appendix:B}. Там я постараюсь раскрыть спин с двух сторон: со стороны чистой математики и со стороны физики. Думаю, его должно быть достаточно для более полного понимания, что же такое этот спин и как с ним обращаться.

Очень упрощая, можно сказать, что спин -- это собственный момент импульса частицы. Это не значит, что она крутится -- просто это экспериментальный факт, который был подтвержден в огромном количестве работ. Стоит воспринимать спин как естественную характеристику частицы.

Итак, теперь общее вращение частицы будет задаваться двумя слагаемыми: $\hat{J} = \hat{L} + \hat{S}$, где $\hat{S}$ - спиновый оператор. Его особенность в том, что он коммутирует с операторами координаты и импульса, а значит никак не действует на них.
\[
\left[\hat{S}_j, \, \hat{x}_k\right] = 0,\; \left[\hat{S}_j, \, \hat{p}_k\right] = 0,\; \left[\hat{S}_j, \, \hat{L}_k\right] = 0.
\]
С самим собой оператор $\hat{S}$ коммутирует так же, как и оператор момента импульса.
\[
\left[\hat{S}_j, \, \hat{S}_k\right] = i\hbar\varepsilon_{jkl}\hat{S}_l
\]
Так же, как и у операторов момента импульса, у состояний спина есть повышающие и понижающие операторы.
\[
\hat{S}_{\pm}\ket{s, m_s} = \sqrt{s(s+1) + m_s(m_s \pm 1)}\ket{s, m_s \pm 1}
\]
И точно так же должны выполняться следующие уравнения:
\[
\hat{S}^2\ket{s,m_s} = \hbar^2 s(s+ 1)\ket{s,m_s}, \quad \hat{S}_z\ket{s,m_s} = \hbar m_s\ket{s,m_s}
\]
Аналогично для проекций на другие оси:
\begin{align*}
    \hat{S}_x & = \frac{\hat{S}_+ + \hat{S}_-}{2};\\
    \hat{S}_y & = \frac{\hat{S}_+ - \hat{S}_-}{2i}
\end{align*}
Соответственно, как и в случае момента импульса, у спина будет проекция на ось Z. Обозначаться она будет $m_s$. Ведёт она себя также, как и m - её значения лежат во множестве {-s, -s + 1, ..., s}. Разница в том, что спин может принимать полуцелые значения -- значит, и проекция может быть полуцелая.

Давайте рассмотрим два самых основных примера -- $S = 0$ и $S = 1/2$.
\begin{itemize}
    \item $S = 0$. Тогда и проекция спина на ось z тоже равна нулю: $S_z = 0$. Тогда $\hat{S} = 0$ и волновые функции будут такими, какими мы их знали до того, как появилась внутренняя степень свободы. Примеры частиц с нулевым спином: Бозон Хиггса, $\pi$-мезон.
    \item $S = 1/2$. Проекция спина на ось теперь имеет две компоненты: $S_z = \{1/2, -1/2\}$. Обозначим состояние спина ``вверх'' (в смысле положительной проекции на z) как $\ket{\uparrow} = \ket{1/2,\, 1/2}$, а состояние ``вниз'' -- $\ket{\downarrow} = \ket{1/2, -1/2}$. Посмотрим, как на эти состояния действуют разные операторы.
    \begin{align*}
        &S_z\ket{\uparrow} = \hbar/2\ket{\uparrow}, \quad\, S_z\ket{\downarrow} = -\hbar/2 \ket{\downarrow}\\
        &S_+\ket{\downarrow} = \hbar\ket{\uparrow},\qquad S_+\ket{\uparrow} = 0\\
        &S_-\ket{\uparrow} = -\hbar\ket{\downarrow},\quad\, S_-\ket{\downarrow} = 0\\
        &S_x\ket{\uparrow} =\hbar/2\ket{\downarrow}, \quad\, S_y\ket{\uparrow} =-i\hbar/2\ket{\downarrow}
    \end{align*}
    Зная, как действуют операторы на состояния вверх и вниз, найдём их матрицы:
    \[
    \hat{S}_i \simeq \left(\begin{array}{cc}
        \bra{\uparrow}\hat{S}_i\ket{\uparrow} & \bra{\uparrow}\hat{S}_i\ket{\downarrow} \\
        \bra{\downarrow}\hat{S}_i\ket{\uparrow}  & \bra{\downarrow}\hat{S}_i\ket{\downarrow}
    \end{array}
    \right) = \frac{\hbar}{2}\hat{\sigma}_i,
    \]
    где $\sigma_i$ - \textit{матрицы Паули}. Поговорим про них подробнее, как только найдём, чему они равны. Итак:
    \begin{align*}
    &S_x = \frac{\hbar}{2}\sigma_x = \frac{\hbar}{2}\left(\begin{array}{cc} 0 & 1\\ 1 & 0 \end{array}\right),\\
    &S_y = \frac{\hbar}{2}\sigma_y = \frac{\hbar}{2}\left(\begin{array}{cc} 0 & -i\\ i & 0 \end{array}\right),\\
    &S_z = \frac{\hbar}{2}\sigma_z = \frac{\hbar}{2}\left(\begin{array}{cc} 1 & 0\\ 0 & -1 \end{array}\right).
    \end{align*}
\end{itemize}
Мы видим, что матрицы Паули -- унитарные и эрмитовые матрицы, которыми можно представить обезразмеренный оператор спина. Давайте решим несколько упражнений, чтобы подробнее разобраться с их свойствами.
\excersize{Упражнение №26}{darklavender}
\begin{center}
    \textit{Докажите справедливость следующих соотношений для матриц Паули:}
    \[
    \hat{\sigma}_k\hat{\sigma}_l = \delta_{kl} + ie_{klm}\hat{\sigma_m}
    \]
    \[
    (\hat{\boldsymbol{\sigma}}\mathbf{a})(\hat{\boldsymbol{\sigma}}\mathbf{b}) = (\mathbf{ab}) + i(\boldsymbol{\sigma}[\mathbf{a} \cross \mathbf{b}])
    \]
\end{center}
Ещё раз выпишем явный вид матриц паули для спина $S=1/2$:
\[
    \hat{\sigma}_x = \left(\begin{array}{cc}
        0 & 1 \\
        1 & 0
    \end{array}
    \right), 
    \hat{\sigma}_y = \left(\begin{array}{cc}
        0 & -i \\
        i & 0
    \end{array}
    \right),
    \hat{\sigma}_z = \left(\begin{array}{cc}
        1 & 0 \\
        0 & -1
    \end{array}
    \right)
\]
Данные матрицы обладают следующими свойствами:
\begin{itemize}
    \item $\hat{\sigma}^2 = \hat{1}$,
    \item $[\hat{\sigma}_k, \hat{\sigma}_l] = 2ie_{klm}\sigma_m$,
    \item $\{\hat{\sigma}_k, \hat{\sigma}_l\} = 2\delta_{kl}$, где $\{\hat{\sigma}_k, \hat{\sigma}_l\} = \hat{\sigma}_k\hat{\sigma}_l + \hat{\sigma}_l\hat{\sigma}_k$ -- антикоммутатор.
\end{itemize}
Пользуясь данными свойствами, можем записать:
\[
\hat{\sigma_k}\hat{\sigma_l} = \frac{1}{2}([\hat{\sigma}_k, \hat{\sigma}_l] + \{\hat{\sigma}_k + \hat{\sigma}_l\}) = \frac{1}{2}(2ie_{klm}\sigma_m + 2\delta_{kl}) = \delta_{kl} + ie_{klm}\hat{\sigma}_m,
\]
\[
(\boldsymbol{\sigma}\mathbf{a})(\boldsymbol{\sigma}\mathbf{b}) = \hat{\sigma}_ka_k\hat{\sigma}_lb_l = \hat{\sigma}_k\hat{\sigma}_la_kb_l = (ie_{klm}\hat{\sigma_m} + \delta_{kl})a_kb_l = ie_{klm}\sigma_ma_kb_l + a_kb_k = i(\hat{\sigma}[\mathbf{a}\cross\mathbf{b}]) + (\mathbf{a}\mathbf{b}).
\]
\csquare{darklavender}


\excersize{Упражнение №27}{darklavender}
\begin{center}
    \textit{Найдите собственные значения и собственные векторы спинового оператора $\hat{\sigma}_n = (\hat{\Vec{\sigma}}, \Vec{n})$, где $\Vec{n}$ -- единичный вектор с составляющими:}
    \[
    \Vec{n} = (\sin\theta\cos\phi, \sin\theta\sin\phi, \cos\theta)
    \]
    \textit{Обсудите случаи, когда вектор $\Vec{n}$ направлен вдоль осей $x$, $y$ и $z$.}
\end{center}
Запишем скалярное произведение вектора матриц паули на вектор $\Vec{n}$:
\begin{align*}
    \hat{\sigma}_n & = (\hat{\Vec{\sigma}}, \Vec{n}) = \hat{\sigma}_x n_x + \hat{\sigma}_y n_y + \hat{\sigma}_z n_z = \left(\begin{array}{cc}
        \cos\theta & \sin\theta(\cos\phi - i\sin\phi) \\
        \sin\theta(\cos\phi + i\sin\phi)  & -\cos\Phi
    \end{array}
    \right) = \\ 
    & = 
    \left(\begin{array}{cc}
        \cos\theta & e^{-i\phi} \sin\theta\\
        e^{i\phi} \sin\theta & -\cos\Phi
    \end{array}
    \right)
\end{align*}
Теперь, зная явный вид оператора $\hat{\Vec{\sigma}}_n$, можем решить характеристическое уравнение и найти собственные значения и векторы:

\begin{align*}
    &|\hat{\Vec{\sigma}}_{n} - \lambda E| = \left|\begin{array}{cc}
        \cos\theta - \lambda & e^{-i\phi}\sin\theta \\
         e^{i\phi} & -\cos\theta - \lambda
    \end{array}
    \right| = -\cos^2\theta + \lambda^2 -\sin^2\theta = \lambda^2 - 1 = 0 \; => 
    \\ & => \lambda = \pm 1
\end{align*}
Соответствующие этим собственным значениям собственные функции (нормированные на единицу) имеют вид:
\[
\Psi_1 = \left(\begin{array}{c}
     \cos \frac{\theta}{2} \\
     e^{i\phi} \sin\frac{\theta}{2}
\end{array}
\right),\;
\Psi_{-1} = \left(\begin{array}{c}
    -sin\frac{\theta}{2} \\
    e^{i\phi}\cos\frac{\theta}{2}
\end{array}
\right)
\]
Вектор-функции, элементы которых определяют состояние спина у частиц, мы будем называть \textit{спинорами}. Линейная комбинация функций $\Psi_1$ и $\Psi_2$ как раз и образует спинор. Запомните это обозначение, в будущем оно нам ещё понадобится.
Рассмотрим случаи, когда $\Vec{n}$ направлен вдоль осей $x$, $y$ и $z$:
\begin{enumerate}
    \item $\Vec{n} = (1, 0, 0) \; => \; \theta = \frac{\pi}{2}, \; \phi = 0.$ Подставляя данные значения в найденные собственные функции, получим $\Psi_1 = \left(\begin{array}{c} \frac{\sqrt{2}}{2} \\ \frac{\sqrt{2}}{2} \end{array} \right),\;
    \Psi_{-1} = \left(\begin{array}{c} -\frac{\sqrt{2}}{2} \\ \frac{\sqrt{2}}{2}\end{array}\right)$
    \item $\Vec{n} = (0, 1, 0) \; => \; \theta = \phi = \frac{\pi}{2}$, откуда $\Psi_1 = \left(\begin{array}{c} \frac{\sqrt{2}}{2} \\ \frac{i\sqrt{2}}{2} \end{array}\right),\;
    \Psi_{-1} = \left(\begin{array}{c} -\frac{\sqrt{2}}{2} \\ \frac{i\sqrt{2}}{2} \end{array}\right)$
    \item $\Vec{n} = (0, 0, 1) \; => \; \theta = 0, \; \phi$ любой, откуда $\Psi_1 = \left(\begin{array}{c} 1 \\ 0 \end{array} \right),\;
    \Psi_{-1} = \left(\begin{array}{c} 0 \\ 1 \end{array}\right)$ 
\end{enumerate}
Последний результат особенно важен, так как с помощью него мы будем обозначать волновые функции введённых выше состояний вверх $\ket{\uparrow}$ и вниз $\ket{\downarrow}$.
\csquare{darklavender}
Теперь, когда мы разобрались с матрицами Паули, можно порешать упражнения про спин.
\excersize{Упражнение №28}{darklavender}
\begin{center}
    \textit{Пусть электрон находится в состоянии с проекцией спина на ось $z$, равной $1/2$. Найдите вероятность того, что проекция спина этого электрона на направление $n$ равна $1/2$ (или $-1/2$).}
\end{center}
Из прошлого упражнения мы знаем, как выглядит волновая функция для спина вверх: $\psi_{1/2} = \left(\begin{array}{c} 1 \\ 0 \end{array}\right)$. Так же мы знаем базисные вектора для электрона, спин которого равен 1/2: $\Psi_1 = \left(\begin{array}{c}  \cos \frac{\theta}{2} \\ e^{i\phi} \sin\frac{\theta}{2}\end{array}\right),\; \Psi_{-1} = \left(\begin{array}{c} -sin\frac{\theta}{2} \\ e^{i\phi}\cos\frac{\theta}{2} \end{array} \right)$. Разложим исходное состояние $\psi_{1/2}$ по этим базисным векторам:
\[
\psi_{1/2} = (\Psi_1^{\dagger}\psi_{1/2})\Psi_1 + (\Psi_{-1}^{\dagger}\psi_{1/2})\Psi_{-1}
\]
Заметим, что $\Psi_1^{+} = (\cos\frac{\theta}{2}, e^{-i\phi}\sin\frac{\theta}{2}), \; \Psi_{-1}^{+} = (-\sin\frac{\theta}{2}, e^{i\phi}\cos\frac{\theta}{2})$, откуда
\begin{align*}
    \psi_{1/2} &= (\cos\frac{\theta}{2}, e^{-i\phi}\sin\frac{\theta}{2}) \left(\begin{array}{c} 1 \\ 0 \end{array}\right)\Psi_1 + (-\sin\frac{\theta}{2}, e^{i\phi}\cos\frac{\theta}{2}) \left(\begin{array}{c}
    1 \\ 0 \end{array}\right)\Psi_{-1} = \\
\\ &= \cos\frac{\theta}{2}\Psi_1 - \sin\frac{\theta}{2}\Psi_{-1}
\end{align*}
Как мы помним, если у нас есть разложение нашего состояния, то его вероятность оказаться в каком-то из состояний есть модуль квадрата коэффициента, стоящего при этом состоянии. Тогда, запишем итоговые вероятности:
\[
    P(S_z = 1/2) = \cos^2\frac{\theta}{2},\quad P(S_z = -1/2) = \sin^2\frac{\theta}{2},
\]
\csquare{darklavender}