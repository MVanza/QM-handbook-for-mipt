\begin{center}
    \section{Семинар II}
\end{center}
\subsection{Коммутатор. Соотношение неопределенностей}
\hspace{1em} На прошлом семинаре мы обсуждали возможность измерения наблюдаемых, используя собственные значения и собственные функции. Это связано с тем, что наблюдаемое имеет нулевую неопределенность: 
\[
\sqrt{\langle\Delta A^2 \rangle} = \sqrt{\bra{\psi}\hat{A}^2\ket{\psi} - \bra{\psi}\hat{A}\ket{\psi}^2} = 0
\]
Вы наверняка помните из курса общей физики про принцип неопределенности Гейзенберга, согласно которому две физические величины могут находиться в таком соотношении, что измерить их одновременно точно невозможно. Так вот оказывается, что, зная операторы этих величин, можно определить, будут они совместны для системы или нет. И для этого нам понадобится новый оператор -- коммутатор. Он записывается в виде квадратной скобки, внутри которой находятся два оператора, и показывает степень коммутации между этими операторами:
\[
[\hat{A}, \hat{B}] = \hat{A}\hat{B} - \hat{B}\hat{A}.
\]
Из выражения очевидно, что, если операторы коммутируют, то коммутатор равен нулю. Иначе он будет равен какому-то другому оператору. Решим небольшое упражнение на коммутаторы.
\excersize{Упражнение №6}{darklavender}
\begin{center}
\textit{Убедитесь в справедливости следующих соотношений:}
\[
[\hat A \hat B, \hat C] = \hat A [\hat B, \hat C] + [\hat A, \hat C]\hat B, \quad [\hat A, \hat B \hat C] = \hat B [\hat A, \hat C] + [\hat A, \hat B] \hat C.
\]
\end{center}

Задача просто на определение коммутатора. Раскроем левую и правую часть:
\[
\hat A \hat B \hat C - \hat C \hat A \hat B = \hat A \hat B \hat C \hcancel{$- \hat A \hat C \hat B + \hat A \hat C \hat B$} - \hat C \hat A \hat B = \hat A \hat B \hat C - \hat C \hat A \hat B
\]
Второе по аналогии.
\csquare{darklavender}

Важная теорема: \textit{Пусть $\hat{A}$ и $\hat{B}$ -- два эрмитовых оператора. Тогда существует базис, в котором они одновременно диагонализируются (у них есть общая система собственных векторов состояний), тогда и только тогда, когда $[\hat{A}, \hat{B}] = 0$}. Идея доказательства следующая:
\begin{enumerate}
    \item Если одни одновременно приводимы к диагональному виду, то достаточно записать их в виде спектрального разложения по общим базисным векторам $\hat{A} = \sum_i A_i \ket{v_i}\bra{v_i}$, $\hat{B} = \sum_i B_i \ket{v_i}\bra{v_i}$ и рассмотреть их произведение.
    \item Если же они коммутируют, то можно показать, что действие одного из операторов на собственный вектор второго само по себе является собственным состоянием: $\hat{A} \ket{a_i} = a_i \ket{a_i} \implies \hat{B} \hat{A} \ket{a_i} = a_i \hat{B} \ket{a_i} \implies \hat{A} ( \hat{B} \ket{a_i} ) = a_i ( \hat{B} \ket{a_i} )$. Далее через пропорциональность $\hat{B} \ket{a_i}$ и $\ket{a_i}$ показываем, что вектор $\ket{a_i}$ собственный и для $\hat{B}$.
\end{enumerate}

Из этого следует, что, если операторы коммутируют, то их можно одновременно точно измерить. Если же нет, то точность измерения будет определяться принципом неопределённости Гейзенберга:
\[
\langle \Delta \hat{A}^2 \rangle \langle \Delta \hat{B}^2 \rangle \geq \frac{1}{4} |\langle [\hat{A}, \; \hat{B}] \rangle|^2
\]
Доказательство общей теоремы я здесь приводить не буду, лишь оставлю два неравенства, которые проще вывести самому. Итоговое неравенство получается из них напрямую:
\begin{gather*}
    |\langle [\hat{A}, \; \hat{B}] \rangle|^2 \leq 4|\langle \hat{A}\hat{B}\rangle|^2 \\
    \langle \hat{A}^2 \rangle \langle \hat{B}^2 \rangle \geq |\langle \hat{A}\hat{B}\rangle|^2
\end{gather*}
Выполним упражнение на неопределенность.
\excersize{Упражнение №7}{darklavender}
\begin{center}
\textit{Пусть $[\hat A, \hat B] = i\hat C$. A, B, C -- эрмитовые операторы. Покажите, что в этом случае они удовлетворяют соотношению неопределенности}
\[
\langle \Delta \hat A^2\rangle \langle \Delta \hat B^2\rangle \geq \frac{1}{4} \langle \hat C\rangle^2, \textit{ где } \Delta \hat A = \hat A - \langle \hat A \rangle.
\]
\end{center}

Для начала разберёмся с дисперсией. Посмотрим на коммутатор дисперсий операторов:
\begin{align*}
    [\Delta \hat A, \Delta \hat B] &= \left( \hat A - \langle \hat A \rangle \right) \left( \hat B - \langle \hat B \rangle \right) - \left( \hat B - \langle \hat B \rangle \right) \left( \hat A - \langle \hat A \rangle \right) =\\
    &= \hat A \hat B - \hat A \langle \hat B \rangle - \langle \hat A \rangle \hat B + \langle \hat A \rangle \langle \hat B \rangle - \hat B \hat A + \hat B \langle \hat A \rangle + \\
    &+ \langle \hat B \rangle \hat A - \langle \hat A \rangle \langle \hat B \rangle = \hat A \hat B - \hat B \hat A = [\hat A, \hat B] = i\hat C.
\end{align*}
Мы сократили все члены со средними значениями, так как среднее значение -- это просто число и оно всегда коммутирует с операторами.

Введём вектор состояния $\ket{\varphi}$, определенный следующим образом: $\ket{\varphi} = (\Delta\hat A - i\gamma\Delta\hat B)\ket{\psi}$, где $\gamma \in \mathbb{R}$. Так как скалярное произведение вектора на самого себя -- величина неотрицательная, т.е. $\bra{\varphi}\ket{\varphi} \geq 0$, подставим значение векторов и перейдём к квадратному уравнению относительно $\gamma$:
\begin{align*}
    \bra{\varphi} \ket{\varphi} &= \bra{\psi} \left( \Delta \hat A + i \gamma \Delta \hat B \right) \left( \Delta \hat A - i \gamma \Delta \hat B \right) \ket{\psi} = \bra{\psi} \Delta \hat{A}^2 \ket{\psi} + \\
    &+ \bra{\psi} i \gamma \Delta \hat{B} \Delta \hat{A} \ket{\psi} - \bra{\psi} i \gamma \Delta \hat{A} \Delta \hat{B} \ket{\psi} + \bra{\psi} \gamma^2 \Delta \hat{B}^2 \ket{\psi} = \\
    &= \langle \Delta \hat A^2 \rangle - i \gamma \langle [\Delta \hat{A}, \; \Delta \hat{B}] \rangle + \gamma^2 \langle \Delta \hat B^2 \rangle = \\
    &= \gamma^2 \langle \Delta \hat B^2 \rangle - \gamma \langle \hat{C} \rangle + \langle \Delta \hat A^2 \rangle \geq 0
\end{align*}
Получилось квадратное уравнение, ветви параболы смотрят вверх. Так как нам нужны решения большие или равные нулю, подходит только дискриминант меньший или равный нулю. То есть
\[
\langle \hat C \rangle^2 - 4 \langle \Delta \hat B^2 \rangle \langle \Delta \hat A^2 \rangle \leq 0 \implies
\langle \Delta \hat A^2 \rangle \langle \Delta \hat B^2 \rangle \geq \frac{1}{4} \langle \hat C \rangle^2.
\]
\csquare{darklavender}
\subsection{Оператор импульса и координаты}
\hspace{1em} Первые две величины, которые приходят на ум после разговора о принципе неопределенностей -- это координата и импульс. Чтобы ввести их операторы, сначала вспомним про переход от дискретного спектра к непрерывному. 
\begin{center}
\begin{tabular}{ |c|m{12em}|m{15em}| } 
 \hline
 & \textbf{Дискретный базис $\ket{a_i}$} & \textbf{Непрерывный базис $\ket{x}$} \\ 
 \hline
 Ортонормальность & $\bra{a_i} \ket{a_j} = \delta_{ij}$ & $\bra{x} \ket{x'} = \delta (x - x')$ \\ 
 \hline
 Разложение состояния & \(\displaystyle \ket{\psi} = \sum_i \psi_i \ket{a_i}\) &  \(\displaystyle \ket{\psi} = \int\limits_{-\infty}^{+\infty} \psi(x) \ket{x}dx\) \newline \(\displaystyle \psi(x) = \bra{x} \ket{\psi}\) \\ 
 \hline
 Измерения & $P(a_i) = |\bra{a_i} \ket{\psi}|^2$ \newline (вероятность) & $P(x) = |\bra{x} \ket{\psi}|^2$\newline (плотность вероятности) \\ 
 \hline
 Разложение оператора & $A_{ij} = \bra{a_i} \hat{A} \ket{a_j}$ \newline \(\displaystyle \hat{A} = \sum_{i,j} A_{ij} \ket{a_i} \bra{a_j}\) & $A(x, x') = \bra{x} \hat{A} \ket{x'}$ \newline \(\displaystyle \hat{A} = \int\limits_{-\infty}^{+\infty} \int\limits_{-\infty}^{+\infty} A(x, x') \ket{x} \bra{x'} dx \; dx'\) \\ 
 \hline
 Разложение $\hat{1}$ & \(\displaystyle \hat{1} = \sum_i \ket{a_i} \bra{a_i} \) & \(\displaystyle \hat{1} = \int\limits_{-\infty}^{+\infty} \ket{a_i} \bra{a_i} dx\) \\ 
 \hline
\end{tabular}
\end{center}

Теперь можно ввести два новых оператора: оператор координаты $\hat{x}$ и оператор импульса $\hat{p}$. Начнём с координаты. В отличие от геометрического пространства, где координата имеет размерность один, гильбертово пространство обладает бесконечной размерностью, т.е. существует бесконечно много координатных собственных состояний $\ket{x}$\footnote[1]{Обычно, если внутри кет состояния пишется символ, совпадающий с оператором, например $\hat{x}\ket{x}$, то таким образом указывают на то, что эти состояния собственные для оператора. То же самое и с $\hat{p}\ket{p}$.}, и все эти собственные состояния ортогональны. Самое важное, что стоит понимать про собственные состояния непрерывных наблюдаемых, это то, что они \textit{нефизичны} -- невозможно поместить частицу в абсолютно точную позицию или заставить её двигаться с абсолютно точной скоростью. Поэтому $\ket{x}$ и $\ket{p}$ представляют собой математическую абстракцию. Все физически реальные состояния есть лишь линейная комбинация этих состояний. Давайте посмотрим, как действует оператор координаты на свои собственные состояния:
\begin{align*}
    \hat{x} \ket{x} &= \left( \int_{-\infty}^{+\infty} x' \ket{x'} \bra{x'}dx' \right) \ket{x} = \int_{-\infty}^{+\infty} x' \ket{x'} \bra{x'} \ket{x} dx' = \\
    &= \int_{-\infty}^{+\infty} x' \ket{x'} \delta(x'-x) dx' = x \ket{x}.
\end{align*}
Получается, что действие оператора координаты на свои собственные состояния сводится к умножению состояния на значение координаты. 

Теперь перейдём к оператору импульса и его собственным состояниям. Для этого постулируем отношение между собственными состояниями координаты и импульса:
\[
\bra{x} \ket{p} = \frac{1}{\sqrt{2 \pi \hbar}}e^{i \frac{px}{\hbar}}
\]
Эта формула утверждает, что волновая функция состояния с определенным значением импульса представляет собой бесконечную волну, известную как волна де Бройля. Эта волна -- проявление корпускулярно-волнового дуализма. Из этой формулы можно вывести следующую связь:
\begin{gather*}
    \ket{p} = \frac{1}{\sqrt{2 \pi \hbar}}\int\limits_{-\infty}^{+\infty} e^{i \frac{px}{\hbar}} \ket{x} dx \\
    \ket{x} = \frac{1}{\sqrt{2\pi\hbar}} \int\limits_{-\infty}^{+\infty} e^{-i \frac{px}{\hbar}} \ket{p} dp
\end{gather*}

Давайте посмотрим на квадрат волновой функции де Бройля: $|\bra{x} \ket{p}|^2 = \frac{1}{2 \pi \hbar}$. Если мы возьмём интеграл от этой константы по всему пространству, то он будет бесконечен. Дело в том, что, как уже говорилось ранее, собственные состояния оператора координаты и импульса нефизичны, так как мы предполагаем, что знаем импульс точно. Реальные же состояния нормированные, и модуль квадрата будет соответствовать вероятности. Давайте разберёмся, откуда у нас появляется этот нормировочный множитель, вычислив $\bra{p}\ket{p'}$ через $\bra{x}\ket{x'}$ (далее я буду опускать пределы интеграла, кроме случаев, где они неочевидны):
\begin{align*}
    \bra{p} \ket{p'} &= \frac{1}{2 \pi \hbar} \int \int e^{i \frac{p'x' - px}{\hbar}} \bra{x} \ket{x'} dx dx' = \\
    &= \frac{1}{2 \pi \hbar} \int \int e^{i \frac{p'x' - px}{\hbar}} \delta(x' - x) dx dx' =\\ 
    &= \frac{1}{2 \pi \hbar}\int e^{i \frac{(p' - p) x}{\hbar}}dx = \frac{1}{2 \pi \hbar} 2 \pi \hbar \delta\left(p - p'\right) = \delta(p - p').
\end{align*}
В предпоследнем выражении мы воспользовались тем, что $\int_{-\infty}^{+\infty} e^{iapx} dx = \frac{2 \pi \delta(p)}{|a|}$ -- это следует из фурье разложения $f(x) = 1$ при $a=1$ и затем обобщается на произвольную $a\neq 0$.

Сейчас, дорогие читатели, начинаем читать внимательно. На этом вопросе очень многие валятся как на сдачах, так и на экзамене. Поэтому читаем и запоминаем, как переходить от координатного представления к импульсному и обратно. Для перехода между представлениями мы будем пользоваться тем самым единичным оператором и его различными представлениями:
\[
\hat{1}_x = \int \ket{x} \bra{x} dx, \; \hat{1}_p = \int \ket{p} \bra{p} dp
\]
Используя его, найдём явную формулу для преобразования координатного представления $\psi(x)$ в $\psi(p)$.
\begin{align*}
    \psi(x) &= \bra{x} \ket{\psi} = \bra{x} \hat{1}_p \ket{\psi} = \bra{x} \left( \int \ket{p} \bra{p} dp \right) \ket{\psi} = \\
    &= \int \bra{x} \ket{p} \bra{p} \ket{\psi} dp = \frac{1}{\sqrt{2 \pi \hbar}} \int e^{i \frac{p x}{\hbar}} \psi(p) dp.
\end{align*}
Обратно то же самое, только с минусом в экспоненте:
\[
\psi(p) = \frac{1}{\sqrt{2 \pi \hbar}} \int e^{-i \frac{p x}{\hbar}} \psi(x) dx.
\]
Часто можно услышать, что переход осуществляется за счёт Фурье преобразования. Это действительно так, когда мы работаем с волновым числом $k=p/\hbar$. Тогда $\psi(p) = \psi(k)/\sqrt{\hbar}$, и переход будет иметь вид $\psi(x) = \frac{1}{\sqrt{2 \pi \hbar}}\int e^{i k x}\psi(k) dk$, что соответствует Фурье преобразованию. 

Легко показать, что оператор импульса действует на свои собственные состояния так же, как и оператор координаты на свои: $\hat{p} \ket{p}=p \ket{p}$. Но как действует оператор импульса в координатном представлении? Чтобы это выяснить, найдём матричный элемент оператора импульса, т.е. $\bra{x} \hat{p} \ket{x'}$:
\[
\bra{x} \hat{p} \ket{x'} = \int p \bra{x} \ket{p} \bra{p} \ket{x'} dp = \frac{1}{2 \pi \hbar} \int p e^{\frac{i p}{\hbar}(x - x')} dp.
\]
Представим $p e^{\frac{i p}{\hbar}(x - x')} = -i \hbar \frac{d}{dx} e^{\frac{i p}{\hbar}(x - x')}$. Тогда
\begin{gather*}
    \frac{1}{2 \pi \hbar} \int p e^{\frac{i p}{\hbar}(x - x')} dp = \frac{1}{2 \pi \hbar}(-i h) \frac{d}{dx} \int e^{\frac{i p}{\hbar}(x - x')} dp = \\
    = \frac{1}{2 \pi \hbar}(-i h)\frac{d}{dx}(2 \pi \hbar)\delta(x - x') = -i \hbar \frac{d}{dx} \delta(x - x')
\end{gather*}
Теперь получим, как действует оператор на произвольное состояние в координатном представлении:
\begin{align*}
    \bra{x} \hat{p} \ket{\psi} &= \bra{x} \hat{p} \hat{1} \ket{\psi} = \int \bra{x} \hat{p} \ket{x'} \bra{x'} \ket{\psi} dx' = \\
    &= -i \hbar \int \left[\frac{d}{dx} \delta(x - x')\right] \psi(x') dx' = \\
    &= -i \hbar \frac{d}{dx} \left[\int \delta(x - x') \psi(x') dx'\right] = -i \hbar \frac{d}{dx} \psi(x).
\end{align*}
Действие оператора импульса на волновую функцию в координатном базисе можно записать как $\hat{p} \psi(x) = -i \hbar \frac{d}{dx} \psi(x)$. Такими же рассуждениями можно получить, что $\hat{x} \psi(p) = i \hbar \frac{d}{dp} \psi(p)$. Это очень важные формулы, которые мы постоянно будем использовать в дальнейшем. Выполним несколько упражнений на закрепление материала.
\excersize{Упражнение №8}{darklavender}
\begin{center}
\textit{Докажите следующие коммутационные соотношения:}
\begin{gather*}
    [\hat x_{\alpha}, F(\hat{\mathbf p})] = i \hbar \frac{\partial F(\hat{\mathbf p})}{\partial \hat p_{\alpha}} \\
    [\hat p_{\alpha}, G(\hat{\mathbf r})] = -i \hbar \frac{\partial G(\hat{\mathbf r})}{\partial \hat x_{\alpha}}
\end{gather*}
\end{center}

Начнём со второго. Мы знаем, как оператор импульса действует на волновую функцию в координатном базисе: $\hat p_{\alpha} \psi(\mathbf r) = -i \hbar \frac{\partial \psi(\mathbf r)}{\partial x_{\alpha}}$. Остаётся только вспомнить, что коммутатор сам по себе тоже является оператором, и применить его к волновой функции в координатном базисе:
\begin{gather*}
    [\hat p_{\alpha}, G(\hat{\mathbf{r}})] f(\mathbf{r}) = -(i\hbar \frac{\partial G(\hat{\mathbf{r}})}{\partial x_{\alpha}}f(\mathbf{r}) + i\hbar G(\hat{\mathbf{r}}) \frac{\partial f(\mathbf{r})}{\partial x_{\alpha}}) + i\hbar G(\hat{\mathbf{r}})\frac{\partial f(\mathbf{r})}{\partial x_{\alpha}} = -i\hbar \frac{\partial G(\hat{\mathbf{r}})}{\partial x_{\alpha}} f(\mathbf{r})
\end{gather*}
Второе делается аналогично, только вместо действия оператора импульса на волновую функцию в координатном представлении, мы действуем оператором координаты на волновую функцию в импульсном представлении. 
\csquare{darklavender}

\excersize{Упражнение №9}{darklavender}
\begin{center}
\textit{Используя ``табличный'' коммутатор операторов импульса и координаты и результаты упражнений 6 и 8, раскройте следующие коммутаторы:}
\[
[\hat x_{\alpha}, \hat{\mathbf{p}}^2], [U(\mathbf{r}), \hat{\mathbf{p}}], [U(r), \hat{\mathbf{p}}^2].
\]
\end{center}

Сначала покажем, чему равен коммутатор $[\hat{x}_{\alpha}, \hat{p}_{\beta}]$:
\begin{gather*}
    [\hat{x}_{\alpha}, \hat{p}_{\beta}] \psi(x) = \hat{x}_{\alpha} \hat{p}_{\beta} \psi(x) - \hat{p}_{\beta} \hat{x}_{\alpha} \psi(x) = \\
    = -x_{\alpha} i \hbar \frac{\partial}{\partial x_{\beta}} \psi(x) + i \hbar \frac{\partial x_{\alpha}}{\partial x_{\beta}} \psi(x) + i \hbar x_{\alpha} \frac{\partial}{\partial x_{\beta}} \psi(x) = i \hbar \delta_{\alpha \beta}\psi(x).
\end{gather*}
Получается, $[\hat{x}_{\alpha}, \hat{p}_{\beta}] = i \hbar \delta_{\alpha \beta}$. Идём дальше, пользуясь результатом, полученном в упражнении 6:
\begin{align*}
[\hat x_{\alpha}, \hat{\mathbf{p}}^2] &= \hat{p}_{\beta} [\hat{x}_{\alpha}, \hat{p}_{\beta}] + [\hat{x}_{\alpha}, \hat{p}_{\beta}] \hat{p}_{\beta} = \\ &= 2 i \hbar \delta_{\alpha \beta} \hat{p}_{\beta} = 2 i \hbar \hat{p}_{\alpha}.
\end{align*}
Следующее ещё проще. Мы знаем выражение для $[\hat p, U(\hat r)]$, здесь нужно найти обратный коммутатор. А он равен "прямому" с другим знаком, то есть
\[
[U(\mathbf{r}), \hat{p}_{\alpha}] = - [\hat{p}_\alpha, U(\mathbf{r})] = i \hbar \nabla U(\mathbf{r})
\]
В последнем придётся немного посчитать, но я не думаю, что вас это волнует, считать же мне придётся...
\begin{gather*}
    [U(r), \hat{\mathbf{p}}^2] = \hat{\mathbf{p}}[U(r), \hat{\mathbf{p}}] + [U(r), \hat{\mathbf{p}}]\hat{\mathbf{p}} = \hat{\mathbf{p}}i\hbar U'(r)\frac{\mathbf{r}}{r} + i\hbar U'(r)\frac{\mathbf{r}}{r}\hat{\mathbf{p}} = \\
    = \hbar^2(U''(r)\frac{\mathbf{r}^2}{r^2} + U'(r) \frac{\nabla\mathbf{r}}{r} + U'(r)\mathbf{r}(-\frac{1}{r^2})\,\frac{\mathbf{r}}{r} + 2U'(r) \frac{\mathbf{r}}{r}\frac{d}{d\mathbf{r}}) = \\
    =\hbar^2(U''(r) + U'(r) \frac{2}{r} + 2U'(r) \frac{\mathbf{r}}{r}\frac{d}{d\mathbf{r}})
\end{gather*}
\csquare{darklavender}

\excersize{Упражнение №10}{darklavender}
\begin{center}
\textit{Используя свойства оператора трансляции и результаты упражнений 8 и 9, преобразуйте операторное выражение:}
\[
e^{\frac{i}{\hbar} a \hat{\mathbf{p}}} U(\mathbf{r}) e^{-\frac{i}{\hbar} a \hat{\mathbf{p}}}.
\]
\end{center}

Поработаем немного с оператором трансляции. Мы уже знаем, что, действуя на функцию, оператор трансляции смещает её на значение a. Давайте попробуем найти связь между $\hat T_{a}$ и $\hat{p}$. Для этого разложим функцию, на которую подействовал оператор трансляции, в ряд Тейлора:
\[
\hat T_{a} f(x) = f(x + a) = \sum\limits_{k=0}^{\infty} \frac{a^k f^{(k)}(x)}{k!} = \sum\limits_{k=0}^{\infty} \frac{a^k (\frac{\partial}{\partial x})^k f(x)}{k!} = e^{a \frac{\partial}{\partial x}} f(x)
\]
Вспомним, что оператор импульса в координатном представлении действует как производная. Тогда, можно записать:
\[
\hat T_{a} = e^{a\nabla} = e^{\frac{i}{h}a\hat{\mathbf{p}}}
\]
Возвращаясь к условию упражнения, мы видим, что это равносильно действию оператора трансляции с двух сторон (и с разными знаками а!)
\[
e^{\frac{i}{\hbar} a \hat{\mathbf{p}}} U(\mathbf{r}) e^{-\frac{i}{\hbar} a \hat{\mathbf{p}}} f(\mathbf{r}) = \hat T_{a} U(\mathbf{r}) \hat T_{-a} f(\mathbf{r}) = \hat T_{a}U(\mathbf{r}) f(\mathbf{r - a}) = U(\mathbf{r + a}) f(\mathbf{r})
\]
\csquare{darklavender}
\excersize{Упражнение №11}{darklavender}
\begin{center}
\textit{Постройте оператор, соответствующий физической величине $\varphi = (\mathbf{rp})$, где $\mathbf r$ и $\mathbf p$ -- соответственно радиус-вектор и импульс частицы.}
\end{center}

Оператор физической величины \textit{всегда} эрмитов. Оператор $\hat{r}\hat{p}$ не эрмитов, так как при эрмитовом сопряжении порядок операторов меняется, а они не коммутируют. Значит, нужно составить из них такую комбинацию, чтобы при эрмитовом сопряжении оператор $\hat{\varphi}$ не менялся даже с учетом изменения порядка $\hat{p} \hat{r}$. На ум приходит такая комбинация:
\[
\hat{\varphi} = \frac{1}{2} (\hat{r} \hat{p} + \hat{p} \hat{r})
\]
\csquare{darklavender}

На следующем семинаре поговорим про основное уравнение динамики квантовой системы -- уравнение Шрёдингера.