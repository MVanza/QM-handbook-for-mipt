\begin{center}
    \section{Благодарности}
\end{center}

Первоначально эти записи подразумевались как конспекты, используя которые мне было бы проще рассказывать семинары. В какой-то момент пришло понимание, что эти записи могут стать полноценным самостоятельным учебным пособием, с погружением в теорию, подробным решением задач и красивыми картинками. Первоначально задача казалась неподъёмной. Оказалось, что это вполне посильно. Всё, что нужно, это люди, которые смогут помочь и поддержать в самые сложные и не очень моменты. Таким людям и посвящена эта глава.

Первым делом хочу сказать огромное спасибо Хамидуллиной Галие. Если вам понравились рисунки, вы не нашли опечатки в формулах или смогли разобраться, как себя ведёт частица в трёхмерном осцилляторе -- это всё её работа. Помимо того, что она хорошо знает математику и может взять любой интеграл, она так же всегда готова помочь, когда это особенно необходимо. Без неё эта работа не получилось бы такой полноценной и самостоятельной. Не говоря уже о том, что она помогала мне и на моих обычных семинарах. Её по праву можно назвать соавтором как минимум этого пособия.

Когда пишешь такую работу, очень важно сохранять интерес и мотивацию. Это было бы невозможно без очень близкого и важного для меня человека -- Койновой Наталии. Её постоянная поддержка и напоминание о том, насколько важна и интересна эта работа помогали мне на протяжении всего этапа написания семинаров. Когда я терял веру в себя, именно она помогала мне её найти и снова сесть за рабочий стол. Даже если бы я написал эту работу сам, без неё она бы никогда не вышла. Помимо психологической поддержки, Наташа идеально знает русский язык. Поэтому, она выступала в роли редактора этих семинаров. Надеюсь, от неё не ускользнула ни одна моя ошибка. Спасибо ей за всё.

Так же хочется сказать спасибо моему самому первому набору: Коле, Глебу, Алёне, Тимуру, Саше и Галие (уже упомянутой выше). Они смогли подарить мне уверенность в том, что эта работа действительно кому-то нужна и может помочь при решении домашнего задания, на сдачах и на экзамене. Самое важное, что они доверились мне, и пошли на ещё совсем сырые семинары. Благодаря им они стали гораздо лучше.

Последнюю благодарность я бы хотел выразить своему преподавателю по статистической физике -- Суровцеву Евгению Владимировичу. И хоть мы встретились с ним, когда я уже начал рассказывать квантовую механику, он послужил для меня преподавательским ориентиром. Его очень доброе, местами ироничное отношение к студентам в совокупности с глубоким знанием предмета и большим желанием его понятно объяснить делают из него показательного лектора и семинариста. Если когда-нибудь я стану преподавателем, именно он будет той планкой, на которую я буду равняться. 