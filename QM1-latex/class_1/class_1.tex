\begin{center}
    \section{Семинар I}
\end{center}
\subsection{Введение}


\hspace{1em} Рассказывая о новом направлении в науке, важно определиться, по какому пути будут проходить первые шаги в изучении ещё не изведанной для слушателя области. Я вижу два основных пути: первый -- вести повествование в хронологическом порядке, второй -- выстраивать его с точки зрения чётких и формальных постулатов и определений. Примеры для первого случая -- курсы по общефизу. Вместо того чтобы сразу дать уравнения Максвелла, нас аккуратно ведут за руку через весь путь, который прошли физики, чтобы прийти к тем самым 4 важнейшим уравнениям. Пример для второго случая -- любая математическая дисциплина. Прямо сейчас на читателя должна нахлынуть волна ностальгии по первому курсу и попытке лектора объяснить ещё совсем школьникам, что же такое эти действительные числа и каким аксиомам они подчиняются.

Мы планируем заниматься физикой, но физикой теоретической! А значит, и подход будет смешанным, т.е. начнём мы со строгих (где это необходимо) математических определений, а затем перейдём к основным достижениям в физике, которые получилось достичь благодаря аппарату квантовой механики. Я считаю этот подход наиболее логичным, так как проработка математического фундамента сразу позволяет сосредоточиться на физике эксперимента, а значит, развить ту самую физическую интуицию.

Стоит уточнить, что данную работу не следует воспринимать как лекции по квантовой механике. С точки зрения теории здесь будет ``смузи'' из всевозможных источников, которые когда-то попали мне в голову и были нещадно поглощены и обработаны. Теория здесь для понимания практики. В идеале это пособие должно научить вас не только решать задачи, но и понимать их суть. Давайте же приступим!

\subsection{Математический аппарат квантовой механики}
\hspace{1em} В квантовой механике описание объектов будет осуществляться в комплексном \textit{гильбертовом пространстве} $\mathcal{H}$. Гильбертово пространство является обобщением евклидового, добавляя в него полноту. Напомню, что полнота -- это свойство, при котором любая фундаментальная последовательность сходится к элементу из этого пространства. Это очень важное свойство, так как мы будем описывать состояние частиц, используя, в том числе, бесконечномерные объекты. Для таких объектов важно, чтобы они не расходились и лежали внутри нашего пространства. Полнота как раз и обеспечивает эти условия. 

Рассмотрим примеры пространств, которые являются Гильбертовыми: 
\begin{itemize}
    \item Комплексное двумерное пространство $\mathbb{C}^2$. С его помощью мы будем описывать, например, частицы со спином 1/2 (электроны, нейтроны).
    \item Комплексное n-мерное пространство $\mathbb{C}^n$. Используя его, мы будем описывать уже n-уровневые системы, такие как квантовый осциллятор.
    \item Пространство квадратично интегрируемых функций $L^2(\mathbb{R})$. Является первым пространством, с которым мы встретимся, так как используется для описания операторов координаты и импульса.
\end{itemize}

\subsubsection*{Вектор состояния. Первый постулат квантовой механики}
\hspace{1em} Основным инструментом описания состояния частицы является элемент Гильбертового пространства, который мы будем называть \textit{кет вектор} $\ket{\psi} \in \mathcal{H}$. Здесь наступает очень важный момент (не то что бы до этого он был не очень важный, но всё же) -- пришло время познакомиться с обозначением Дирака, или \textit{Bra-ket} нотацией! Помимо того, что она будет использоваться при изучении теории, сложно переоценить вклад данного подхода в упрощение записи различных задач и в обеспечение удобства их понимания. Итак, мы будем описывать состояния частиц с помощью кет векторов, т.е. используя правую угловую скобку ($\ket{\psi}, \; \ket{\varphi}$ и т.п.). Кет -- это элемент Гильбертового пространства, он может быть как конечномерным, так и бесконечномерным. Кет векторы можно складывать, умножать на скаляр, делать что угодно, что позволяет линейная алгебра. В том числе, мы можем делать преобразования с использованием операторов. Мы обязательно вернёмся к этому чуть позже. 

У кет вектора есть ``злой брат-близнец'' -- бра. Хоть бра и называют вектором, на самом деле его правильно воспринимать как функционал, который при скалярном произведении с кет вектором даёт скаляр. Другими словами, если кет векторы лежат в пространстве $\mathcal{H}$, то бра векторы лежат в дуальном пространстве $\mathcal{H}^{dual}$. Учитывая, что бра будет обозначаться левой угловой скобкой, т.е. $\bra{\psi}$, при действии бра на кет получается скалярное произведение, которое можно записать следующим образом: $\bra{\psi}\ket{\psi} \in \mathbb{C}$. 

Теперь у нас есть вся информация, необходимая для формулировки первого постулата квантовой механики: \textit{Состояние изолированной физической системы в фиксированный момент времени задается вектором состояния (кет вектором) $\ket{\psi}$, который принадлежит Гильбертовому пространству $\mathcal{H}$, называемому пространством состояний.}

Рассмотрим конкретные векторы состояний квантового объекта, чтобы стало чуть понятнее. Возьмём самый очевидный квантовый объект -- фотон (частица света). У фотона несколько характеристик, а значит, мы можем по-разному задать базис для векторов его состояний. Очень иллюстративный пример -- рассмотреть поляризацию. Напомню, что вектор поляризации указывает, куда направлен вектор E при распространении света. Обозначим вертикальную поляризацию через $\ket{V}$. Тогда ортогональной к вертикальной будет, очевидно, горизонтальная $\ket{H}$. Так как они ортогональные, можно с уверенностью написать $\bra{H}\ket{V} = 0$. Однако это не единственный вариант, так как мы могли задать базис через диагональную $\ket{D}$ и антидиагональную $\ket{A}$ поляризацию. В следующем параграфе мы посмотрим, как сделать переход от одного состояния к другому.

\subsubsection*{Операторы. Второй постулат квантовой механики}

\hspace{1em} Отлично, состояния ввели, но что делать с их динамикой? И, тем более, каким образом мы хотим измерять состояние в мире действительных чисел, когда состояние объекта описывается комплексным вектором? Для этого мы воспользуемся понятием линейного оператора. Напомню, что линейный оператор отображает линейное пространство в себя и удовлетворяет аксиомам линейности. Давайте рассмотрим подробнее использование операторов для изменения состояния и измерения.

В линейной алгебре для перехода от одного состояния к другому используются линейные операторы. Вы наверняка заметили, что квантовая механика с линейной алгеброй имеет очень прочную связь. Таким образом, переход от одного состояния к другому будет осуществляться через действие на это состояние неким линейным оператором. Формальная запись этого действия следующая: $\hat{A} \ket{\psi} = \ket{\psi'}$, где $\hat{A}$ -- линейный оператор. Для дальнейшей работы нам понадобятся несколько свойств операторов, а именно:
\begin{itemize}
    \item \textit{Эрмитово сопряженным} оператором $\hat{A^{\dagger}}$ называется оператор, для которого выполняется условие $(Ax, y) = (x, A^{\dagger}y)$. Оператор называется \textit{эрмитовым}, если $\hat{A^{\dagger}} = \hat{A}$.
    \item Оператор называется \textit{унитарным}, если выполняется условие $\hat{A}^{\dagger} = \hat{A}^{-1}$.
    \item Операторы, как и матрицы, необязательно коммутируют: $\hat{A}\hat{B} \neq \hat{B}\hat{A}$. Свойство коммутации операторов играет большую роль в квантовой механике, и мы обязательно рассмотрим его подробнее.
\end{itemize}
Операторы всегда действует направо, т.е. запись $\bra{\varphi}\hat{A}\ket{\psi}$ подразумевает действие на кет вектор $\ket{\psi}$. Чтобы оператор подействовал на бра, нужно воспользоваться эрмитовым сопряжением $(\hat{A}\ket{\psi})^{\dagger} = \bra{\psi}\hat{A}^{\dagger}$. Тут важно отметить, что мы пользуемся тем фактом, что бра лежит в дуальном векторном пространстве и получается путём применения на кет эрмитового сопряжения.

Теперь обсудим измерения с использованием операторов. Для этого вспомним, что такое собственные векторы и собственные значения. Собственный вектор $\ket{a}$ и собственное значение $a$ оператора $\hat{A}$ удовлетворяют следующему условию: $\hat{A}\ket{a} = a\ket{a}$. Если оператор эрмитов, его собственные значения действительные, а собственные векторы можно привести к ортонормированному виду. Доказательство этого утверждения можно легко найти в соответствующей литературе. Сопоставим любому физическому измерению \textit{наблюдаемый оператор} или просто наблюдаемую $\hat{A} = \sum_i a_i \ket{a_i}\bra{a_i}$, где  $\ket{a_i}$ -- базис возможных состоянии системы, а коэффициент $a_i$ -- собственные значения оператора. Такое представление оператора называется \textit{спектральным разложением} Произведение вида $\ket{\psi}\bra{\varphi}$ называется внешним произведением. Внешнее произведение можно представлять себе следующим образом: если векторы $\ket{\psi}$ и $\ket{\varphi}$ имеют одинаковую размерность, то $\ket{\psi}\bra{\varphi}$ -- это матрица, образованная обычным матричным умножением столбца на строку (именно в таком порядке). Оказывается, что эрмитовые операторы всегда можно представить в таком виде.

Самое время сформулировать первую часть второго постулата квантовой механики: \textit{Каждая измеряемая физическая величина A описывается эрмитовым оператором $\hat{A}$, действующим в пространстве состояний $\mathcal{H}$. Этот оператор называется наблюдаемым. Результат измерения физической величины A должен быть одним из собственных значений соответствующей наблюдаемой величины A.}

До этого мы говорили преимущественно о математике и почти не затрагивали физическую сторону вопроса. Теперь, когда мы говорим об измерениях, обойти физику никак не получится. Ведь именно здесь мы первый раз встречаемся с принципиальным отличием классического представления от квантового. При взаимодействии с квантовыми объектами мы возмущаем их состояние. Значит при измерении они потеряют то состояние, в котором находились, и перейдут в новое. Как теоретически описать такие измерения? 

Давайте вернёмся примеру с поляризованным фотоном. Пусть мы не знаем его поляризацию изначально, т.е. его состояние $\ket{\psi}$. Если на пути фотона мы поставим beam splitter, который будет отражать вертикальную поляризацию и пропускать горизонтальную, то мы ожидаем, что фотон либо пройдёт, либо отразится. Однако на деле оказывается, что фотон имеет свойства волны (тот самый корпускулярно-волновой дуализм) и при прохождении через beam splitter фотон будет находиться на обоих путях одновременно. Затем, если мы поставим детектор на обоих путях, то мы будем с равной вероятностью детектировать фотоны с вертикальной и горизонтальной поляризацией. 

Раз у нас появилось слово вероятность, давайте определим, каким образом мы будем её вычислять. Чтобы посчитать вероятность, будем проецировать вектор состояния на одно из возможных состояний объекта. Так, например, если фотон был в состоянии $\ket{\psi}$, то вероятность найти его в состоянии с горизонтальной поляризацией будет $|\bra{H}\ket{\psi}|^2$. Модуль в квадрате мы берём, так как величина в общем случае комплексная, а нам важно получить действительное значение. Тогда, если состояние после прохождения beam splitter было диагональным, т.е. $\ket{D} = \frac{1}{\sqrt{2}}(\ket{H} + \ket{V})$, то вероятность найти фотон в горизонтальном состоянии будет
\[
P(H) = |\bra{H}\ket{D}|^2 = \left|\frac{1}{\sqrt{2}} (\bra{H}\ket{H} + \bra{H}\ket{V})\right|^2 = \frac{1}{2}(1 + 0) = \frac{1}{2}.
\]
Как и ожидалось, вероятность получить горизонтальную поляризацию составляет 50\%.

Вернёмся к математике и сформулируем вторую часть второго постулата: \textit{Когда физическая величина A измеряется для нормированного состояния $\ket{\psi}$, вероятность получения собственного значения $a_i$ задается квадратом амплитуды проекции на соответствующий собственный вектор (правило Борна)}
\[
P(a_i) = |\bra{a_i}\ket{\psi}|^2.
\]

Также нам понадобится среднее значение какой-то наблюдаемой. Напомню, что в общем случае среднее значение есть сумма произведения вероятностей на значение величины. Тогда среднее значение наблюдаемой будет равно
\[
\langle A \rangle = \sum_i a_i |\bra{a_i}\ket{\psi}|^2 = \sum_i a_i |\bra{\psi}\ket{a_i}\bra{a_i}\ket{\psi}| = \bra{\psi}\hat{A}\ket{\psi}
\]
\subsubsection*{Волновая функция}
\hspace{1em} Все выводы в предыдущем параграфе были сделаны из предположения, что наша система дискретная. Давайте обобщим результат на непрерывный случай. Это важно, так как большинство физических величин непрерывные (например, координата или импульс). Для описания базисных состояний будем использовать дельта-функцию (или функцию Дирака). Тогда скалярное произведение запишется как $\bra{x'}\ket{x} = \delta(x - x')$. Напомню основные свойства дельта-функции:
\begin{enumerate}
    \item $\int\limits_{-\infty}^{+\infty}\delta(x)dx = 1$
    \item $\int\limits^{+\infty}_{-\infty}f(x)\delta(x-a)dx = f(a)$
    \item $\int\limits^{+\infty}_{-\infty}\delta(\alpha x)dx = \frac{1}{\alpha}$
\end{enumerate}

В случае непрерывного спектра удобнее работать с \textit{волновой функцией}. Вообще, если следовать классическому курсу Физтеха или книге Ландау-Лифшица, то описание квантовой механики начинается именно с волновой функции. Я не считаю это целесообразным, так как волновая функция на данном этапе развития квантовой механики является скорее объектом, полученным из постулата о состояниях в Гильбертовом пространстве, а не постулатом сама по себе. Заканчивая лирику, напишем определение волновой функции:
\[
\psi(\alpha) = \bra{\alpha}\ket{\psi}.
\]
Волновая функция координаты будет выглядеть как $\psi(x) = \bra{x}\ket{\psi}$, где $\ket{x}$ -- собственные векторы для оператора координаты. Тогда состояние квантового объекта можно описать следующим образом:
\[
\ket{\psi} = \int\ket{x}\psi(x)dx
\]
Из определения несложно заметить, что квадрат модуля волновой функции есть вероятность найти частицу в определенном состоянии непрерывного спектра. Напомню, что скалярное произведение на непрерывном спектре задаётся следующим образом:
\[
(\varphi, \psi) = \int \varphi^*(x) \psi(x) dx
\]
\newpage
Мы сформулировали два постулата квантовой механики и построили формальную нотацию для теоретического описания. Давайте попробуем применить наши знания для решения первых упражнений.
\excersize{Упражнение №1}{darklavender}
\begin{center}
\textit{Найдите операторы, эрмитово сопряженные и обратные по отношению к операторам:\\ а) инверсии $\hat I$ и б) трансляции $\hat T_a$}
\end{center}
Для начала определим, что значат эти операторы. В целом, их действие понятно из названия: оператор инверсии меняет знак аргумента, а трансляция перемещает его. Я думаю, с обратными уже становится понятно, но давайте пойдём по порядку. Сначала найдём сопряженные и обратные для инверсии, потом для трансляции.

\begin{enumerate}
\item $\hat I \psi(x) = \psi(-x)$. Основной способ найти эрмитово сопряженный оператор -- это проверить скалярное произведение. Давайте это и сделаем:
\begin{align*}
    (\varphi, \hat{I}\psi)  &= \int\limits^{+\infty}_{-\infty} \varphi(x) \psi(-x) dx = -\int\limits^{-\infty}_{+\infty} \varphi(-x) \psi(x) dx = \\ 
    &= \int\limits^{+\infty}_{-\infty}\varphi(-x) \psi(x) dx = (\hat{I}\varphi(x), \psi(x)) \implies \hat I^{\dagger} = \hat I
\end{align*}
Первое действие -- меняем $x$ на $-x$. Далее меняем пределы интегрирования, чтобы избавиться от минуса. Возвращаемся к скалярному произведению и видим, что оно равно эрмитово сопряженному, а значит, оператор инверсии эрмитов.

\item Давайте посмотрим, что будет, если дважды применить оператор инверсии к волновой функции:
\[
\hat I^2 \psi(x) = \psi(x) \implies \hat I^2 = \hat 1 \implies \hat{I} = \hat{I}^{-1}
\]
Получается оператор инверсии равен эрмитово сопряженному и обратному. Значит, он не только эрмитов, но ещё и унитарен.

\item $\hat T_a \psi(x) = \psi(x+a)$. Решение аналогичное:
\[
(\phi, \hat{T_a} \psi) = \int\limits^{+\infty}_{-\infty} \phi(x)\psi(x+a) dx = \int\limits^{+\infty}_{-\infty} \phi(x-a) \psi(x)dx = (\hat{T}_{-a}\phi, \psi) \implies \hat T^{\dagger}_a = \hat T_{-a} 
\]
Заметим, что в этот раз оператор оказался не эрмитов, а значит, мы не можем использовать его для описания какой-то физической величины.

\item Тем же самым способом находим:
\[
\hat T_a \hat T_{-a} \psi(x) = \psi(x) \implies \hat T_a^{-1} = \hat T_{-a}
\]
Получается, что оператор трансляции унитарный, но не эрмитовый.
\end{enumerate}
\csquare{darklavender}
\newpage
\excersize{Упражнение №2}{darklavender}
\begin{center}
\textit{Найдите собственные значения и собственные состояния оператора инверсии $\hat I$}
\end{center}
Напомню, что собственные значения $\lambda$ и собственные функции $\varphi$ определяются как \\
$\hat I \varphi = \lambda \varphi$. Домножим левую часть на $\hat I$ и получим
\[
\begin{cases}
\hat I^2 \varphi = \hat I(\lambda \varphi) = \lambda^2 \varphi \\
\hat I^2 \varphi = \varphi
\end{cases}
\implies \lambda^2 = 1 \implies \lambda = \pm 1
\]
Таким образом, собственные значения $\lambda$ равны $\pm 1$. Теперь найдём общий вид собственных функций из следующих рассуждений:
\begin{align*}
    \lambda = 1 &: \hat I \varphi(r) = \varphi(-r) = \varphi(r) \implies \varphi(r) - \text{чётная} \\
    \lambda = -1 &: \hat I \varphi(r) = \varphi(-r) = -\varphi(r) \implies \varphi(r) - \text{нечётная}
\end{align*}
То есть, в зависимости от собственного значения, собственные функции либо чётные, либо нечётные.
\csquare{darklavender}
\excersize{Упражнение №3}{darklavender}
\begin{center}
\textit{Найдите собственные значения и собственные состояния оператора трансляции $\hat T_a$}
\end{center}
Для решения этой задачи сделаем несколько предположений. Рассмотрим действие двух операторов трансляции на волновую функцию:
\[
\hat{T}_{a}\hat{T}_{b}\psi(x) = \psi(x + a + b) = \hat{T}_{a + b}\psi(x)
\]
Тогда, рассматривая уравнение $\hat T_a \varphi(r) = \lambda(a) \varphi(r)$, можно заметить, что
\[
\hat{T}_{a}\hat{T}_{b}\varphi(x) = \lambda(a)\lambda(b)\varphi(x) = \lambda(a + b) \varphi(x).
\]
Значит, нужно подобрать такое значение $\lambda(a)$, чтобы оно удовлетворяло следующему условию: $\lambda(a)\lambda(b) = \lambda(a + b)$. При этом $\lambda(a)$ должно быть комплексным, так как оператор трансляции унитарен, но не эрмитов. Представим $\lambda(a)$ в виде $\lambda(a) = e^{sa}$, где $s \in \mathbb{C}$.
Так как квадрат волновой функции есть вероятность найти частицу в определенном месте пространства, должно выполняться условие нормировки, а именно $\int |\varphi(x)|^2 dx = 1$. Но тогда 
\[
\int |\hat{T}_a\varphi(x)|^2 dx = |\lambda(a)|^2\int |\varphi(x)|^2 dx = 1 \implies |\lambda(a)|^2 = 1.
\]
Этому условию удовлетворяет $s = ik$, где k -- действительное число. Тогда уравнение на собственные значения будет выглядеть так:
\[
\hat{T}_{a}\varphi(x) = e^{ika}\varphi(x).
\]
Осталось подобрать волновую функцию. Понятно, что функция должна быть периодична с периодом $a$. Тогда остаётся только подобрать множитель так, чтобы при действии оператора трансляции он оставался таким же, но ещё домножался на $\lambda(a) = e^{ika}$. Учитывая эти предпосылки, выберем волновую функцию в виде $\varphi_k(x) = e^{ikx}u_k(x)$, где $u_k(x + a) = u_k(x)$. Подействуем оператором трансляции на эту волновую функцию и проверим, что мы сделали правильное предположение:
\[
\hat T_a \varphi_k(x) = e^{ik(x + a)} u_k(x + a) = e^{ika} e^{ikx} u_k(x) = \lambda(a) \varphi_k(x)
\]
Всё сошлось!
\csquare{darklavender}
\newpage
\excersize{Упражнение №4}{darklavender}
\begin{center}
\textit{Найти явный вид оператора $e^{i\varphi\hat I}$}.
\end{center}
Такой оператор называется функцией оператора. Функции оператора оказываются полезными для описания динамики системы. Поговорим об этом подробнее, когда дойдём до уравнения Шрёдингера. Пока вспомним, что, когда мы видим матричную экспоненту, первое, что мы хотим сделать -- разложить в ряд Тейлора. Проверим, что получится:
\begin{gather*}
    e^{i\varphi\hat I} = \sum\limits_{k=0}^{\infty} \frac{(i\varphi\hat I)^k}{k!} = \sum\limits_{s=0}^{\infty} \left(\frac{(i\varphi)^{2s}}{2s!}\right) \hat 1 + \sum\limits_{s=0}^{\infty} \frac{(i\varphi\hat I)^{2s+1}}{(2s+1)!} = \\
    =\hat 1 \sum\limits_{s=0}^{\infty} \frac{(-1)^s\varphi^{2s}}{2s!} + i\hat I \sum\limits_{s=0}^{\infty} \frac{(-1)^{s}\varphi^{2s+1}}{(2s+1)!} = \hat 1 cos\varphi + i\hat I sin \varphi.
\end{gather*}
Оператор $\hat{1}$ называется единичным оператором (вообще, его обозначают как $\hat{I}$, но в нашем случае эта буква занята). Его действие очевидно: $\hat{1}\ket{\psi} = \ket{\psi}$. В спектральном разложении он будет выглядеть как $\hat{1} = \sum_n \ket{n}\bra{n}$. Он оказывается очень полезным, когда нам нужно перейти от одного представления к другому. Потом обязательно вспомним про него. А пока радуемся, что решили ещё одно упражнение.
\csquare{darklavender}
\excersize{Упражнение №5}{darklavender}
\begin{center}
\textit{Эрмитов оператор с дискретным спектром $\hat f(\lambda)$ зависит от параметра $\lambda$. Собственные значения $f_n(\lambda)$ и собственные векторы $\ket{n(\lambda)}$ этого оператора также зависят от $\lambda$. Докажите теорему:}
\[
\frac{\partial f_n(\lambda)}{\partial \lambda} = \bra{n(\lambda)} \frac{\partial \hat f(\lambda)}{\partial \lambda} \ket{n(\lambda)}.
\]
\end{center}

Из условия составим уравнение на собственные значения:
\[
\hat f(\lambda)\ket{n(\lambda)} = f_n(\lambda)\ket{n(\lambda)}
\]
Продифференцируем его правую и левую часть:
\[
\frac{\partial \hat f}{\partial \lambda}\ket{n} + \hat f\frac{\partial}{\partial \lambda}\ket{n} = \frac{\partial f_n}{\partial \lambda}\ket{n} + f_n\frac{\partial}{\partial \lambda}\ket{n}
\]
Умножим на бра вектор $\bra{n}$ и преобразуем:
\[
\bra{n}\frac{\partial \hat f}{\partial \lambda}\ket{n} + \bra{n}\hat f\frac{\partial}{\partial \lambda}\ket{n} = \frac{\partial f_n}{\partial \lambda}\braket{n} + f_n\bra{n}\frac{\partial}{\partial \lambda}\ket{n}
\]
Посмотрим на второе слагаемое. Вспоминая, что, по условию, оператор $\hat f$ эрмитов и действуя им на бра вектор $\bra{n}$, получим 4 слагаемое:
\[
\bra{n}\hat f\frac{\partial}{\partial \lambda}\ket{n} = f_n\bra{n}\frac{\partial}{\partial \lambda}\ket{n}
\]
Сокращая их, получим уравнение из условия.
\csquare{darklavender}
На следующем семинаре поговорим о важном свойстве операторов -- их коммутации. Также обсудим два очень важных оператора: оператор импульса и оператор координаты.