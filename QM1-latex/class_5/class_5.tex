\begin{center}
    \section{Семинар V}
\end{center}
\excersize{Упражнение №17}{darklavender}
\begin{center}
    \textit{Получите стационарное уравнение Шрёдингера из упражнения 14 в импульсном представлении. Найдите решение этого уравнения, описывающее связанное состояние частицы. Преобразуйте найденную волновую функцию частицы в импульсном представлении в волновую функцию в координатном представлении и удостоверьтесь в правильности ответа.}
\end{center}

Вообще, задача очень нудная, но если прорешать её самостоятельно (что я вам очень советую), то можно очень хорошо разобраться с переходами между импульсными и координатными представлениями. Напомню, что в бракет форме уравнение Шрёдингера выглядит так:
\[
i\hbar \frac{d}{dt} \ket{\psi} = \hat H \ket{\psi}
\]
Обычно, мы записываем это уравнение, проецируя вектор состояния на координату. В данном случае нужно спроецировать на вектор импульса. Буду писать эту часть достаточно подробно:
\[
\bra{p} i\hbar \frac{d}{dt} \ket{\psi} = \bra{p} \hat H \ket{\psi}
\]
Раскроем оператор Гамильтона как $\hat H = \frac{\hat p^2}{2m} + U(x)$ и перепишем $\bra{p}\ket{\psi}$ в виде $\psi_p$
\[
i\hbar \frac{d}{dt} \psi_p = \frac{p^2}{2m}\psi_p + \bra{p}U(x)\hat{1}_p\ket{\psi} = \frac{p^2}{2m}\psi_p + \int\bra{p}U(x)\ket{p'}\psi_{p'}\;dp'.
\]
Нам необходимо вычислить матричный элемент $U(x)$ в импульсном представлении. Используем единичный оператор (только на этот раз координаты), вставляя его в наше скалярное произведение:
\begin{gather*}
\int\bra{p}U(x)\hat{1}_x\ket{p'}\psi_{p'}\;dp' = \int \bra{p}U(x)\left[\int\ket{x}\bra{x}dr\right]\ket{p'} \psi_{p'}\;dp'= \\
= \int\int U(x)\bra{p}\ket{x}\bra{x}\ket{p'} \psi_{p'} dx\,dp' = \frac{1}{2\pi\hbar}\int\int U(x) e^{-\frac{i}{h}(p-p')x}\psi_{p'} dx\,dp'
\end{gather*}
Подставляя его в изначальное уравнение, получим уравнение Шрёдингера в импульсном представлении:
\[
i\hbar \frac{d}{dt} \psi_p =\frac{p^2}{2m}\psi_p + \frac{1}{2\pi\hbar}\int\int U(x) e^{-\frac{i}{h}(p-p')x}\psi_{p'} dx\,dp'.
\]
Подставляя потенциальную энергию из упражнения, получим:
\[
\frac{p^2}{2m}\psi_p - \frac{\kappa_0\hbar}{2\pi m}\int\int \delta(x) e^{-\frac{i}{h}(p-p')x}\psi_{p'} dx\,dp'= \frac{p^2}{2m}\psi_p - \frac{\kappa_0\hbar}{2\pi m}\int\psi_{p'} dp' =  E\psi_p
\]
Предположим, что интеграл сходится, и введём для него обозначение $I = \int \psi_{p'}dp'$. Далее, выражая энергию через $\kappa$, получим простое алгебраическое выражение на $\psi_p$:
\[
\frac{p^2}{2m}\psi_p - I\frac{\kappa_0\hbar}{2\pi m} = -\frac{\hbar^2\kappa^2}{2m}\psi_p \implies \psi_p = \frac{\hbar\kappa_0}{\pi}\frac{I}{p^2 + \hbar^2\kappa^2}
\]
Теперь выразим $\psi_p$ обратно через $I$ и получим:
\[
I = \frac{\hbar\kappa_0}{\pi}\int\frac{I}{p^2 + \hbar^2\kappa^2}dp = \frac{\hbar \kappa_0}{\pi}\frac{I \pi}{\hbar\kappa^2} = I\frac{\kappa_0}{\kappa} \implies \kappa_0 = \kappa
\]
Тогда, можем записать энергию как 
\[
E = -\frac{\kappa^2_0\hbar^2}{2m}
\]
Волновую функцию в импульсном представлении можно найти двумя путями: либо решить ранее полученное алгебраическое уравнение относительно $I$ с условием нормировки $\int|\psi_p|^2dp = 1$, либо спроецировав уже известное нам решение $\psi(x) =\sqrt{\kappa_0}e^{-\kappa_0|x|}$ из упражнения 14 на вектор импульса, то есть посчитать скалярное произведение $\bra{p}\ket{\psi}$ (не забыв по дороге воспользоваться единичным проекционным оператором).
\csquare{darklavender}
\excersize{Упражнение №18}{darklavender}
\begin{center}
    \textit{Частица массы m свободно движется вдоль оси $x$ с энергий $E$ и в области $x>0$ попадает в область действия потенциала, который имеет вид:}
    
    \textit{а) прямоугольного потенциальной ямы ширины $a$ и глубины $U_0$,}

    \textit{б) прямоугольного потенциального барьера ширины $a$ и высоты $U_0$.}
    
    \textit{Найдите коэффициенты прохождения $T(E)$ и отражения $R(E)$ частицы от указанных потенциалов.}
\end{center}

Начнём решать эту задачу, как люди последовательные, с условия а, то есть с потенциалом в виде ямы. Запишем уравнения для волновых функций вне ямы:
\[
\begin{cases}
\psi_I = Ae^{i\kappa x} + Be^{-i\kappa x} \\
\psi_{III} = Ce^{i\kappa x} + De^{-i\kappa x}
\end{cases}
\]
Так как нам интересно посмотреть на коэффициенты прохождения и отражения, можем сразу заменить некоторые элементы. В первом уравнении $A = 1$, так как мы предполагаем, что справа приходит полная волна, а $B = r$, где r -- это reflection или коэффициент отражения. Во втором же уравнении мы не рассматриваем ситуацию, когда волна идёт справа, то есть $D = 0$, и $C = t$, где t -- это transmission или коэффициент прохождения. Тогда, система преобразуется к следующему виду:
\[
\begin{cases}
\psi_I = e^{i\kappa x} + re^{-i\kappa x} \\
\psi_{III} = te^{i\kappa x}
\end{cases}
\]

Далее, запишем уравнение для области в потенциальной яме:
\[
\psi_{II}'' + k^2\psi_{II} = 0, \quad k^2 = \frac{2m}{\hbar^2}(|U_0| + E)
\]
Выпишем решение этого уравнения:
\[
\psi_{II}(x) = c_1e^{ikx} + c_2e^{-ikx}
\]
Отмечу, что мы поменяли обозначения $k$, так как теперь рассматриваем положительные энергии (потенциал по модулю, так как мы рассматриваем яму). Воспользуемся условиями непрерывности волновой функции и её производной и найдём коэффициенты:
\[
\begin{cases}
    \psi_I(0) = \psi_{II}(0) \\
    \psi'_I(0) = \psi'_{II}(0) 
\end{cases}
\quad
\begin{cases}
    \psi_{II}(a) = \psi_{III}(a) \\
    \psi'_{II}(a) = \psi'_{III}(a) 
\end{cases}
\]
\[
\begin{cases}
    1 + r = c_1 + c_2 \\
    1-r = \frac{k}{\kappa}(c_1 - c_2) 
\end{cases}
\quad
\begin{cases}
    c_1e^{ika} + c_2e^{-ika} = te^{i\kappa a}\\
    c_1e^{ika} - c_2e^{-ika} = \frac{\kappa}{k}te^{i\kappa a}
\end{cases}
\]
Из правой системы выразим $c_1(t)$ и $c_2(t)$:
\begin{align*}
c_1 &= \frac{t}{2}e^{ia(\kappa - k)}(1 + \frac{\kappa}{k})\\
c_2 &= \frac{t}{2}e^{ia(\kappa + k)}(1 - \frac{\kappa}{k})
\end{align*}
Подставим в левую систему, просуммировав уравнения, и выразим $t$:
\begin{gather*}
2 = \frac{c_1(\kappa + k) + c_2 (\kappa - k)}{\kappa}  \implies \\ 
\implies \frac{1}{t} = \frac{e^{ia(\kappa - k)}(1 + \frac{\kappa}{k})(\kappa + k) + e^{ia(\kappa + k)}(1 - \frac{\kappa}{k}) (\kappa - k) }{4\kappa} \implies \\
\implies t = \frac{4\kappa k}{e^{ia(\kappa - k)}(\kappa + k)^2 - e^{ia(\kappa + k)}(\kappa - k)^2 }
\end{gather*}
Таким же образом можно получить и r:
\[
r = \frac{(\kappa^2 - k^2)\sin (ak)}{2i\kappa k\cos(ak) + (\kappa^2 + k^2)\sin (ak)}
\]
Как мы помним из правила Борна, вероятность определяется как $|\psi|^2$. Тогда, коэффициент прохождения T выражается следующим образом:
\[
T = |\psi_{III}|^2 = |t|^2
\]
Подставив t и сделав несколько преобразований, получим:
\[
T = \frac{1}{1 + \frac{U_0^2}{4E(|U_0| + E)}\sin^2 (ak)}
\]
Будем считать, что всё, что не прошло через барьер, отразилось обратно. Тогда $R = 1 - T$. Из уравнения на $T$ видно, что вероятность частице пролететь над ямой не равна 1. Это значит, что квантовая частица может отразиться от ямы и полететь обратно с вероятностью $1 - T$. Естественно, для классического тела такая ситуация не происходит.

Для случая потенциального барьера решение будет точно такое же, только вместо $ k^2 = \frac{2m}{\hbar^2}(|U_0| + E)$ мы берём $k^2 = \frac{2m}{\hbar^2}(U_0 - E)$, так как считаем, что барьер выше энергии нашей частицы. В итоге получим те же самые уравнения, но с новым k. И в данном случае наблюдается обратная ситуация. Вместо того чтобы отразиться, частица может пролететь сквозь барьер и вылететь с другой стороны. Это называется \textit{туннелированием}. Тоже чисто квантовый эффект.
\csquare{darklavender}
\excersize{Упражнение №19}{darklavender}
\begin{center}
    \textit{Частица массы m находится в связанном состоянии в $\delta$-потенциале (см. упражнение 14). В момент t=0 происходит мгновенное изменение параметра ямы от $\kappa_0$ до $\kappa_1$. Найдите вероятность "ионизации". Обсудите эволюцию волновой функции частицы сразу после ионизации в случае, когда $\kappa_1=0$.}
\end{center}

Расчёты в задаче не представляют никакой сложности, если понимать, что есть что. Разберём по порядку. Мгновенное изменение говорит о том, что уравнение Шрёдингера остаётся стационарным. Вероятность ионизации – вероятность вылета частицы из ямы. Теперь давайте всё покажем наглядно. Из упражнения 14 мы знаем, что волновая функция дельта ямы есть $\psi_0(x) = \sqrt{\kappa_0}e^{-\kappa_0 |x|}$. Соответственно, в момент $t=0$ волновая функция запишется в виде $\psi_1(x) = \sqrt{\kappa_1}e^{-\kappa_1 |x|}$. Найдём вероятность перехода от $\psi_0$ к $\psi_1$, используя правила Борна (т.е. через скалярное произведение):
\begin{align*}
P &= \left|\int\limits_{-\infty}^{+\infty}\psi_0(x)\psi_1(x)dx\right|^2 = \kappa_0\kappa_1\left|\int\limits_{-\infty}^{+\infty} e^{-(\kappa_0 + \kappa_1)|x|}dx\right|^2 =\\
&= \kappa_0\kappa_1\left|2\int\limits_{0}^{+\infty} e^{-(\kappa_0 + \kappa_1)x}dx\right|^2 = \kappa_0\kappa_1\frac{4}{(\kappa_0 + \kappa_1)^2}
\end{align*}
Тогда вероятность выхода, то есть ионизации, есть $1 - P$, то есть
\[
P_{exit} = 1 - \kappa_0\kappa_1\frac{4}{(\kappa_0 + \kappa_1)^2}
\]
\csquare{darklavender}

На этой задаче мы временно прощаемся с одномерными потенциалами и переходим к темам осциллятора и углового момента, которые будут обсуждаться на следующем семинаре. 