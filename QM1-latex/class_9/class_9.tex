\begin{center}
    \section{Семинар IX}
\end{center}
\subsection{Атом водорода}
\hspace{1em} Давайте применим полученные нами знания на практике. Попробуем решить радиальное уравнение для атома водорода. В атоме водорода электрон движется в электростатическом потенциале, создаваемом тяжелым ядром. Потенциал имеет следующий вид:
\[
V(r) = \frac{e^2}{r}, \text{где $e$ - заряд электрона}.
\]
Видим, что потенциал зависит только от $r$. Это значит, что задача об атоме водорода представляет собой частный случай движения в центральном поле. Из предыдущих семинаров мы знаем как выглядит волновая функция этой частицы:
\[
\Psi_{nlm}(r, \theta, \phi) = R_{nl}(r)Y^m_l(\theta, \phi).
\]
Что значит первый индекс n -- выясним в этом семинаре. Угловая часть нам известна, осталось найти только радиальную часть. Значит, надо решить радиальное уравнение:
\[
\left[ -\frac{\hbar^2}{2Mr^2}\frac{\partial}{\partial r}(r^2 \frac{\partial}{\partial r}) + \frac{\hbar^2 l(l+1)}{2Mr^2} - \frac{e^2}{r}\right]R_{nl}(r) = ER_{nl}(r)
\]
Сделаем замену переменной $R_{nl}=U_{nl}(r)/r$ и перепишем уравнение через новую переменную:
\begin{align*}
\frac{1}{r^2}\frac{\partial}{\partial r}\left[ r^2 \frac{\partial}{\partial r} R_{nl} \right] &= \frac{1}{r^2}\frac{\partial}{\partial r}\left[ r^2 \frac{\partial}{\partial r} \frac{U_{nl}}{r} \right] = \frac{1}{r^2}\frac{\partial}{\partial r}\left[ r^2 \left(\frac{U'_{nl}}{r} - \frac{U_{nl}}{r^2}\right) \right] = \frac{U''_{nl}}{r} \implies \\
\implies  &\left[ -\frac{\hbar^2}{2M}\frac{\partial^2}{\partial r^2} + \frac{\hbar^2 l(l+1)}{2Mr^2} - \frac{e^2}{r}\right]U_{nl}(r) = EU_{nl}(r)
\end{align*}

Давайте рассмотрим асимптотики данного уравнения при $r \rightarrow 0$ и $r \rightarrow +\infty$. При $r \rightarrow 0$ доминируют члены с минимальными степенями (максимальными в знаменателе), то есть второй член в квадратных скобках. Перепишем уравнение, оставив только его и производную:
\[
\frac{\partial^2}{\partial r^2} U_{nl}(r) = \frac{l(l+1)}{r^2} U_{nl}(r)
\]
Это уравнение Коши-Эйлера (решается заменой $U_{nl} = r^\alpha$), его решением будет
\[
U_{nl}(r) = c_1r^{l+1} + c_2 r^{-l}
\]
Второе слагаемое отпадает, чтобы волновая функция не имела разрыва при $r = 0$. Представим эту функцию в виде ряда:
\[
U_{nl} = \sum\limits_{j=l+1}^{n}A_j r^j e^{-\kappa r},
\]
где $n$ - некоторое натуральное число и $A_{j+1} \neq 0$ - действительные коэффициенты. 

Найдём коэффициент $\kappa$, рассмотрев этот ряд и радиальное уравнение при $r \rightarrow +\infty$. В это случае доминировать будут члены с максимальной степенью, т.е. правая часть уравнения. Найдём производную ряда:
\[
\frac{\partial^2}{\partial r^2} U_{nl}(r) =  \sum\limits_{j=l+1}^{n} A_j[\kappa^2r^j - 2\kappa jr^{j-1} + j(j-1)r^{j-2}]e^{-\kappa r}.
\]
Тогда, подставляя максимальную степень, получим:
\[
 -\frac{\hbar^2}{2M}A_n\kappa^2r^n e^{-\kappa r} = E A_n r^n e^{-\kappa r}
\]
Отсюда $\kappa = \sqrt{-2ME}/\hbar$.

Теперь найдём коэффициенты $A_j$ и $n$. Для этого подставим в радиальное уравнение функцию $U_{nl}(r)$ в виде ряда, умножим обе стороны на $2M/\hbar^2$ и выразим энергию через $E = -\hbar^2\kappa^2/2M$. Тогда уравнение преобразуется к следующему виду:
\begin{align*}
    2\kappa\sum\limits_{j = l+1}^n jA_j r^{j-1} - \sum\limits_{j = l+1}^n j(j-1)A_j r^{j-2} + l(l+1)\sum\limits_{j = l+1}^n A_j r^{j-2} - \frac{2Me^2}{\hbar^2}\sum\limits_{j = l+1}^n jA_j r^{j-1} =0
\end{align*}
Сгруппируем подобные члены и перепишем выражение:
\[
\sum\limits_{j = l+1}^n\left(2\kappa j - \frac{2Me^2}{\hbar^2}\right)A_j r^{j-1} + \sum\limits_{j = l+1}^n [l(l+1) - j(j-1)] A_j r^{j-2} =0
\]
Если во второй сумме сделать замену на $j' = j + 1$, можно заметить, что при $j'=l$ выражение внутри суммы равно нулю. Значит, суммирование мы можем начать с $j'= l+ 1$. Тогда можно переписать:
\[
\sum\limits_{j = l+1}^n\left[\left(2\kappa j - \frac{2Me^2}{\hbar^2}\right)A_j+ [l(l+1) - j(j+1)] A_{j+1} \right]r^{j-1} = 0
\]
Этот многочлен равен нулю при любой степени $r$ тогда, когда выполняется рекурсивное соотношение
\[
\left(2\kappa j - \frac{2}{a}\right)A_j+ [l(l+1) - j(j+1)] A_{j+1} = 0
\]
Здесь мы ввели новую важную величину $a = \frac{\hbar^2}{Me^2} \sim 0.53 \text{\r{A}}$ - \textit{боровский радиус}. Его физический смысл будет понятен чуть позже.

С коэффициентами $A$ разобрались, осталось найти условие на $n$. Можно появиться вопрос: почему мы вообще решили, что сумма должна быть конечна? Оказывается, что получившийся ряд расходится, если $n=\infty$. Действительно, при больших $j$ $A_j \sim (2\kappa)^j/j!$, и тогда функция имеет вид
\[
U_{nl} \sim \sum\limits_j\frac{(2\kappa r)^j}{j!}e^{-\kappa r} \rightarrow e^{2\kappa r}e^{-\kappa r} = e^{\kappa r}
\]
Волновая функция, которая стремится к бесконечности, нефизична. Значит, $n$ должно быть меньше бесконечности. Условие того, что сумма конечна, выполняется, если множитель перед $A_j$ обнуляется при некотором $j = n$. В этом случае
\[
2\kappa n = \frac{2}{a}
\]
и все $A_j$ при $j > n$ обнуляются.

Теперь мы обладаем достаточным количеством знаний, чтобы найти радиальные волновые функции для конкретных значений $n$ и $l$. Попробуем это сделать в виде упражнения.
\newpage
\excersize{Упражнение №28}{darklavender}
\begin{center}
    \textit{Вычислите радиальные волновые функции $R_{nl}(r)$ атома водорода при $ a)\;n = 1,\, l =0$, $b)\;n = 2,\, l =0$ и $c)\;n = 2,\, l =1$}
\end{center}

a)  При $n = 1,\, l=0$ получим равенство $\kappa = 1/a$. Из всех коэффициентов $A_j$ ненулевым остаётся только $A_1$, так как значения $j$ лежат в промежутке от $l+1$ до $n$. Тогда, вспоминая, что $R_{nl}(r) = U_{nl}(r)/r$, получаем
\[
R_{10}(r) = A_1e^{-r/a}
\]
Получим значение коэффициента $A_1$ из нормировки волновой функции:
\begin{align*}
\int\limits_0^{+\infty}|R_{10}(r)|^2 r^2 dr = & A^2_1\int\limits_0^{+\infty}e^{-2r/a}r^2 dr = A^2_1 2!\left(\frac{a}{2}\right)^3 = \frac{a^3}{4}A^2_1 = 1 \implies \\ 
&\implies A_1 = 2a^{-3/2}
\end{align*}
Тогда итоговая радиальная часть волновой функции имеет вид
\[
R_{10}(r) = 2a^{-3/2}e^{-r/a}
\]

b) При $n = 2,\, l=0$ получим $\kappa = \frac{1}{2a}$. В этот раз не будут обнуляться два коэффициента -- $A_1$ и $A_2$. В нашем случае они связаны соотношением
\[
\left[ 2\kappa - \frac{2}{a} \right]A_1 - 2A_2 = 0 \implies A_2 = -\frac{A_1}{2a}
\]
Тогда радиальная часть волновой функции имеет вид
\[
R_{20}(r) = A_1\left( 1 - \frac{r}{2a} \right) e^{-r/2a}
\]
Отнормируем:
\begin{gather*}
\int\limits_0^{+\infty}|R_{20}(r)|^2 r^2 dr = A^2_1\int\limits_0^{+\infty}\left( r^2 - \frac{r^3}{a} +\frac{r^4}{4a^2} \right)e^{-r/a} dr = \\ 
= A^2_1 a^3 (2! - 3! + 4!/4) = 2A^2_1 a^3 = 1 \implies \\ 
\implies A_1 = (2a^3)^{-1/2}
\end{gather*}
Итоговый вид:
\[
R_{20}(r) = (2a^3)^{-1/2}\left( 1 - \frac{r}{2a} \right) e^{-r/2a}
\]
c) Для последнего случая $n=2,\, l=1$ у нас остаётся только коэффициент $A_2$, радиальная волновая функция равна
\[
R_{21}(r) = A_2re^{-r/2a}
\]
Нормируя, получим
\begin{align*}    
\int\limits_0^{+\infty}|R_{21}(r)|^2 r^2 dr = & A^2_2\int\limits_0^{+\infty}r^4 e^{-r/a} dr = 4! A^2_2 a^5 = 1 \implies \\ \implies & A_2 = (24a^5)^{-1/2}
\end{align*}
Итог:
\[
R_{21}(r) = (24a^5)^{-1/2}re^{-r/2a}
\]
Посмотрев на получившиеся радиальные волновые функции, становится чуть понятнее, в чём физический смысл боровского радиуса: он определяет характерный размер волновых функций, а также примерный радиус орбитали основного состояния.
\csquare{darklavender}

Теперь определим энергетический спектр атома водорода. У нас для этого уже всё готово -- достаточно соединить два уравнения на $\kappa$ (выраженное через $n$ и через $E$):
\[
E_n = -\frac{1}{2M}\left(\frac{\hbar}{na}\right)^2 = -\frac{M}{2}\left(\frac{e^2}{n\hbar}\right)^2
\]
Заметим, что энергия от $l$ не зависит, только от $n$. Поэтому $n$ называется \textit{главным квантовым числом}. Оно определяет степень вырождения по остальным квантовым числам. 

Давайте вспомним все квантовые числа, которые у нас имеются на данный момент:
\begin{itemize}
    \item $n \in \mathbb{N}$ - главное квантовое число
    \item $l \in {0, 1, 2, ..., n-1}$ - орбитальное квантовое число
    \item $m \in {-l, -l+1, ... , l-1, l}$ - магнитное квантовое число.
\end{itemize}
Получается, что при заданном главном квантовом числе $n$, вырождение определяется как $2n^2$. Действительно, количество значений $l$ равно $n$, степень вырождения $l$ в свою очередь $2l+1$. Тогда, посчитав сумму первых $n$ членов арифметической прогрессии, получим $\sum\limits_{l=0}^{n-1}(2l +1 ) = n^2$. Двойка появится, если мы учтём вырождение по спину: он может принимать два значения -- $\pm 1/2$.

\excersize{Упражнение №29}{darklavender}
\begin{center}
    \textit{Атом водорода находится в состоянии с главным квантовым числом $n = 2$. Вычислите все возможные проекции дипольного момента атома $\mathbf{d} = \bra{\psi}e\hat{r}_j\ket{\psi'}$}
\end{center}

Дипольный момент характеризует переход атома из одного состояния в другое после излучения фотона. Значит, остаться на одном и том же уровне электрон никак не может -- если фотон излучился, то энергия должна уменьшится. Давайте покажем это на одной из проекций. Напомню, что $r_j = \begin{pmatrix}
    r\sin\theta\cos\phi & r\sin\theta\sin\phi & r\cos\theta
\end{pmatrix}^T$, для $x$, $y$ и $z$ соответственно.
\begin{gather*}
    \bra{2,\, 0,\, 0}z\ket{2,\, 0,\, 0} = \int\limits_{0}^{+\infty}\int\limits_{0}^{\pi}\int\limits_{0}^{2\pi}(R_{20})^2(Y^0_0)^2 r\cos\theta \, r^2\sin\theta \, dr\, d\theta\, d\phi =\\ = \frac{1}{8a^3\pi} \int\limits_{0}^{+\infty}\int\limits_{0}^{\pi}\int\limits_{0}^{2\pi} \left( 1 - \frac{r}{2a} \right)^2 e^{-r/a} r^3\cos\theta\sin\theta\, dr \, d\theta\, d\phi = \\  = \left(\int\limits_{0}^{\pi} \cos\theta \sin\theta d\theta = 0\right) = 0
\end{gather*}
То же самое будет и со всеми остальными проекциями состояния на самих себя. 

Теперь посмотрим на переходы между энергетическими уровнями. Давайте попробуем сделать это в общем виде:
\begin{align*}
    \bra{nlm}\begin{pmatrix} x \\ y \\ z \end{pmatrix}\ket{n'l'm'} &= \int\limits_{0}^{+\infty} r^2 R^*_{nl} R_{n'l'} dr \int\limits_{0}^{\pi}\sin\theta\,d\theta\int\limits_{0}^{2\pi}(Y^{m'}_{l'}) (Y^m_l)^* \begin{pmatrix} r\sin\theta\cos\phi \\ r\sin\theta\sin\phi \\ r\cos\theta \end{pmatrix} d\phi = \\ & = I_r(n,\,l,\,n',\,l')I_{\theta\phi}(l,\,m,\,l',\,m'),
\end{align*}
где $I_r(n,\,l,\,n',\,l')$ и $I_{\theta\phi}(l,\,m,\,l',\,m')$ - радиальный и угловой интеграл, соответственно:
\begin{gather*}
I_r(n,\,l,\,n',\,l') = \int\limits_{0}^{+\infty} r^3 R^*_{nl} R_{n'l'} dr\\
I_{\theta\phi}(l,\,m,\,l',\,m') = \int\limits_{0}^{\pi}\sin\theta\,d\theta\int\limits_{0}^{2\pi}(Y^{n'}_{l'}) (Y^n_l)^* \begin{pmatrix} \sin\theta\cos\phi \\ \sin\theta\sin\phi \\ \cos\theta \end{pmatrix} d\phi
\end{gather*}
Посчитаем сначала радиальные интегралы, так как их всего два.
\begin{align*}
    I_r(1,\,0,\,2,\,0) &= \int\limits_{0}^{+\infty} R_{10}^*\,R_{20}\,r^3\,dr = \int\limits_{0}^{+\infty} 2a^{-\frac{3}{2}} e^{-\frac{r}{a}} \, (2a)^{-\frac{1}{2}} \left(1 - \frac{r}{2a}\right) e^{-\frac{r}{2a}} \, r^3 dr = \\
    & = \frac{\sqrt{2}}{a^3} \int\limits_{0}^{+\infty} e^{-\frac{3}{2a}r} r^3 dr - \frac{1}{\sqrt{2}a^4} \int\limits_{0}^{+\infty} e^{-\frac{3}{2a}} r^4 dr = \\
    & = \left(\int\limits_{0}^{+\infty} x^n e^{-ax} dx = \frac{n!}{a^{n + 1}}, n = 0, 1, 2..., a > 0\right) = -\frac{32\sqrt{2}}{81} a
\end{align*}
\[
    I_r(1, 0, 2, 1) = \int\limits_{0}^{+\infty} R_{10}^* R_{21} r^3 dr = \int\limits_{0}^{+\infty} 2a^{-\frac{3}{2}} e^{-\frac{r}{a}} (24a^5)^{-\frac{1}{2}} r e^{-\frac{r}{2a}} r^3 dr = \frac{256}{81\sqrt{6}} a
\]
Теперь посчитаем угловые интегралы. Начнём с того, что убедимся, что переход между уровнем $\bra{1,\,0,\,0}\mathbf{r}\ket{2,\,0,\,0}$ невозможен.
\begin{align*}
    I_{\theta\phi}(0,\,0,\,0,\,0) &= \int\limits_{0}^{\pi}\sin\theta\,d\theta\int\limits_{0}^{2\pi}(Y^0_0)^2 \begin{pmatrix} \sin\theta\cos\phi \\ \sin\theta\sin\phi \\ \cos\theta \end{pmatrix} d\phi = \\& = \int\limits_{0}^{\pi} \begin{pmatrix} \sin^2 \theta \\ \sin^2 \theta \\ \sin\theta\cos\theta \end{pmatrix} \, d\theta \int\limits_{0}^{2\pi} \left(\frac{1}{2\sqrt{\pi}}\right)^2 \begin{pmatrix} \cos\phi \\ \sin\phi \\ 1 \end{pmatrix} d\phi = \\& = \left( \int\limits_{0}^{2\pi} \cos\phi d\phi = \int\limits_{0}^{2\pi} \sin\phi d\phi = \int\limits_{0}^{\pi} \sin\theta \cos\theta d\theta = 0 \right) = \begin{pmatrix} 0 \\ 0 \\ 0
    \end{pmatrix}
\end{align*}

Найдём переходы, которые могут происходить:
\begin{align*}
    I_{\theta\phi}(0,\,0,\,1,\,0) &= \int\limits_{0}^{\pi}\sin\theta\,d\theta\int\limits_{0}^{2\pi}(Y^0_1)(Y^0_0)^*  \begin{pmatrix} \sin\theta\cos\phi \\ \sin\theta\sin\phi \\ \cos\theta \end{pmatrix} d\phi = \\
    & = \frac{\sqrt{3}}{4\pi}\int\limits_{0}^{\pi} \begin{pmatrix} \sin^3\theta\\ \sin^3 \theta \\ \sin^2\theta\cos\theta \end{pmatrix} d\theta \int\limits_{0}^{2\pi} \begin{pmatrix} \cos\phi \\ \sin\phi \\ 1 \end{pmatrix} d\phi = \\
    & = \frac{\sqrt{3}}{4\pi}\begin{pmatrix} 0 \\  0 \\ 2/3 \end{pmatrix} \begin{pmatrix} 0 \\  0 \\ 2\pi \end{pmatrix} = \frac{\sqrt{3}}{3}\begin{pmatrix} 0 \\  0 \\ 1 \end{pmatrix} \\ 
    \\
    I_{\theta\phi}(0,\,0,\,1,\,1) &= \int\limits_{0}^{\pi}\sin\theta\,d\theta\int\limits_{0}^{2\pi}(Y^1_1)(Y^0_0)^*  \begin{pmatrix} \sin\theta\cos\phi \\ \sin\theta\sin\phi \\ \cos\theta \end{pmatrix} d\phi = \\
    & = -\frac{\sqrt{3}}{4\pi\sqrt{2}}\int\limits_{0}^{\pi} \begin{pmatrix} \sin^2 \theta\cos\theta \\ \sin^2 \theta\cos\theta \\ \sin\theta\cos^2\theta \end{pmatrix} \, d\theta \int\limits_{0}^{2\pi} \begin{pmatrix} e^{i\phi}\cos\phi \\ e^{i\phi}\sin\phi \\ e^{i\phi} \end{pmatrix} \, d\phi = \\
    & = -\frac{\sqrt{3}}{4\pi\sqrt{2}}\begin{pmatrix} 4/3 \\  4/3 \\ 0 \end{pmatrix} \begin{pmatrix} \pi \\ i\pi \\ 0 \end{pmatrix} = -\frac{1}{\sqrt{6}}\begin{pmatrix} 1 \\  i \\ 0 \end{pmatrix} \\
    \\
    I_{\theta\phi}(0,\,0,\,1,\,-1) &= \int\limits_{0}^{\pi}\sin\theta\,d\theta\int\limits_{0}^{2\pi}(Y^0_0)^* \, (Y^{-1}_1) \begin{pmatrix} \sin\theta\cos\phi \\ \sin\theta\sin\phi \\ \cos\theta \end{pmatrix} d\phi = \\& = \frac{\sqrt{3}}{4\sqrt{2}\pi} \int\limits_0^{\pi} \begin{pmatrix} \sin^3 \theta \\ \sin^3\theta \\ \sin^2\theta\cos\theta \end{pmatrix} d\theta \int\limits_{0}^{2\pi} \begin{pmatrix} e^{-i\phi}\cos\phi \\ e^{-i\phi}\sin\phi \\ e^{-i\phi} \end{pmatrix} \, d\phi = \\& = \frac{\sqrt{3}}{4\sqrt{2}\pi} \begin{pmatrix} 4/3 \\ 4/3 \\ 0 \end{pmatrix} \begin{pmatrix} \pi \\ -i\pi \\ 0 \end{pmatrix} = \frac{1}{\sqrt{6}} \begin{pmatrix} 1 \\ -i \\ 0 \end{pmatrix}
\end{align*}
В итоге, получим:
\begin{align*}
 \bra{1,\, 0,\, 0}&x\ket{2,\, 1,\, \pm 1} = \mp \frac{2^7}{3^5}a\\
 \bra{1,\, 0,\, 0}&y\ket{2,\, 1,\, \pm 1} = \mp i\frac{2^7}{3^5}a\\
 \bra{1,\, 0,\, 0}&y\ket{2,\, 1,\,  0} = \frac{2^7 \sqrt{2}}{3^5}a
\end{align*}
\csquare{darklavender}